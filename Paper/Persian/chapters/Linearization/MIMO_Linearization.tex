%\subsection{فرم خطی فضای حالت چهارپره}\label{lin_MIMO}
%در این قسمت با توجه به معادلات فضای حالت به دست آمده، چهارپره حول نقطه کار خطی‌سازی می‌شود. به این منظور، نقطه کار به‌صورت زیر در نظر گرفته شده‌است.
%\begin{equation}
%	\boldsymbol{x^*} = \begin{bmatrix} % bold or vec???????????/
%		0& 0 & 0 & 0& 0& 0
%	\end{bmatrix}^\mathrm{T}
%\end{equation}
%\begin{equation}
%	\boldsymbol{\omega^*} = \begin{bmatrix}
%		2000&2000&2000&2000
%	\end{bmatrix}^\mathrm{T}\rm{RPM}
%\end{equation}
%که $\boldsymbol{x^*}$ بردار حالت تعادلی و $\boldsymbol{\omega^*}$ بردار ورودی حالت تعادلی است.
%%\begin{equation*}
%%	\sigma_1 = \omega_2^2-\omega_4^2,\quad \sigma_2 = \omega_1^2-\omega_3^2,
%%	\quad \sigma_3 = \omega_1^2-\omega_2^2+\omega_3^2-\omega_4^2,\quad \sigma_4 = \omega_1-\omega_2+\omega_3-\omega_4
%%\end{equation*}
%%\begin{equation*}
%%		\boldsymbol a = \begin{bmatrix}
%%%		x_4 + x_5\sin(x_1)\tan(x_2) + x_6\cos(x_1)\tan(x_2)\\
%%%		x_5\cos(x_1)- x_6\sin(x_1)\\
%%%		(x_5\sin(x_1) + x_6\cos(x_1))\sec(x_2)\\
%%%		A_1\cos(x_2)\sin(x_1) + 
%%%		A_2x_5x_6 + A_3\left(\omega_2^2-\omega_4^2\right)+
%%%		A_4x_5\left(\omega_1-\omega_2+\omega_3-\omega_4\right)- \dfrac{x_4}{\lvert x_4\rvert}A_5+A_6\cos(x_1)\\
%%%		B_1\sin(x_2) + 
%%%		B_2x_4x_6 + B_3\left(\omega_1^2-\omega_3^2\right)+
%%%		B_4x_4\left(\omega_1-\omega_2+\omega_3-\omega_4\right)- \dfrac{x_5}{\lvert x_5\rvert}B_5 + B_6\cos(x_2)\\
%%%		C_1x_4x_5 + 
%%%		C_2\left(\omega_1^2-\omega_2^2+\omega_3^2-\omega_4^2\right)- \dfrac{x_6}{\lvert x_6\rvert}C_3
%%		x_4 + x_5\sin(x_1)\tan(x_2) + x_6\cos(x_1)\tan(x_2)\\
%%		x_5\cos(x_1)- x_6\sin(x_1)\\
%%		(x_5\sin(x_1) + x_6\cos(x_1))\sec(x_2)\\
%%		A_1\cos(x_2)\sin(x_1) + 
%%		A_2x_5x_6 + A_3\sigma_1+
%%		A_4x_5\sigma_4- \dfrac{x_4}{\lvert x_4\rvert}A_5+A_6\cos(x_1)\\
%%		B_1\sin(x_2) + 
%%		B_2x_4x_6 + B_3\sigma_2+
%%		B_4x_4\sigma_4- \dfrac{x_5}{\lvert x_5\rvert}B_5 + B_6\cos(x_2)\\
%%		C_1x_4x_5 + 
%%		C_2\sigma_3- \dfrac{x_6}{\lvert x_6\rvert}C_3
%%	\end{bmatrix}
%%\end{equation*} 
%%\begin{equation}
%%	\boldsymbol{x} = \begin{bmatrix} % bold or vec???????????/
%%		\phi& \theta & \psi & p& q& r
%%	\end{bmatrix}^\mathrm{T}
%%\end{equation}
%%\begin{equation}
%%	\boldsymbol{\omega} = \begin{bmatrix}
%%		\omega_1&\omega_2&\omega_3&\omega_4
%%	\end{bmatrix}^\mathrm{T}
%%\end{equation}
%برای خطی سازی از بسط تیلور استفاده شده‌‌است.
%\begin{equation}
%	\delta \dot{\boldsymbol{x}} = \boldsymbol A\delta \boldsymbol x + \boldsymbol B \delta \boldsymbol \omega 
%\end{equation}
%که:
%%\begin{equation}
%%	\delta \dot{\boldsymbol{x}} =
%%	\begin{bmatrix}
%%		\delta \dot x_1&
%%		\delta \dot x_2&
%%		\delta \dot x_3&
%%		\delta \dot x_4&
%%		\delta \dot x_5&
%%		\delta \dot x_6
%%	\end{bmatrix}^\mathrm{T}
%%\end{equation}
%\begin{equation}
%	\boldsymbol A = \left.\dfrac{\partial \boldsymbol f}{\partial  \boldsymbol x}\right\vert_{\boldsymbol{x^*}} 
%\end{equation}
%%	\begin{bmatrix}
%%\dfrac{\partial  a_1}{\partial  x_1}&
%%\dfrac{\partial  a_1}{\partial  x_2}&
%%\dfrac{\partial  a_1}{\partial  x_3}&
%%\dfrac{\partial  a_1}{\partial  x_4}&
%%\dfrac{\partial  a_1}{\partial  x_5}&
%%\dfrac{\partial  a_1}{\partial  x_6}
%%\\[1em]
%%\dfrac{\partial  a_2}{\partial  x_1}&
%%\dfrac{\partial  a_2}{\partial  x_2}&
%%\dfrac{\partial  a_2}{\partial  x_3}&
%%\dfrac{\partial  a_2}{\partial  x_4}&
%%\dfrac{\partial  a_2}{\partial  x_5}&
%%\dfrac{\partial  a_2}{\partial  x_6}
%%\\[1em]
%%\dfrac{\partial  a_3}{\partial  x_1}&
%%\dfrac{\partial  a_3}{\partial  x_2}&
%%\dfrac{\partial  a_3}{\partial  x_3}&
%%\dfrac{\partial  a_3}{\partial  x_4}&
%%\dfrac{\partial  a_3}{\partial  x_5}&
%%\dfrac{\partial  a_3}{\partial  x_6}
%%\\[1em]
%%\dfrac{\partial  a_4}{\partial  x_1}&
%%\dfrac{\partial  a_4}{\partial  x_2}&
%%\dfrac{\partial  a_4}{\partial  x_3}&
%%\dfrac{\partial  a_4}{\partial  x_4}&
%%\dfrac{\partial  a_4}{\partial  x_5}&
%%\dfrac{\partial  a_4}{\partial  x_6}
%%\\[1em]
%%\dfrac{\partial  a_5}{\partial  x_1}&
%%\dfrac{\partial  a_5}{\partial  x_2}&
%%\dfrac{\partial  a_5}{\partial  x_3}&
%%\dfrac{\partial  a_5}{\partial  x_4}&
%%\dfrac{\partial  a_5}{\partial  x_5}&
%%\dfrac{\partial  a_5}{\partial  x_6}
%%\\[1em]
%%\dfrac{\partial  a_6}{\partial  x_1}&
%%\dfrac{\partial  a_6}{\partial  x_2}&
%%\dfrac{\partial  a_6}{\partial  x_3}&
%%\dfrac{\partial  a_6}{\partial  x_4}&
%%\dfrac{\partial  a_6}{\partial  x_5}&
%%\dfrac{\partial  a_6}{\partial  x_6}
%%	\end{bmatrix}
%
%\begin{equation}
%	\boldsymbol B = \left.\dfrac{\partial \boldsymbol f}{\partial \boldsymbol \omega}\right\vert_{\boldsymbol{\omega^*}}
%%	\begin{bmatrix}
%%		\dfrac{\partial  a_1}{\partial  \omega_1}&
%%		\dfrac{\partial  a_1}{\partial  \omega_2}&
%%		\dfrac{\partial  a_1}{\partial  \omega_3}&
%%		\dfrac{\partial  a_1}{\partial  \omega_4}
%%		\\[1em]
%%		\dfrac{\partial  a_2}{\partial  \omega_1}&
%%		\dfrac{\partial  a_2}{\partial  \omega_2}&
%%		\dfrac{\partial  a_2}{\partial  \omega_3}&
%%		\dfrac{\partial  a_2}{\partial  \omega_4}
%%		\\[1em]
%%		\dfrac{\partial  a_3}{\partial  \omega_1}&
%%		\dfrac{\partial  a_3}{\partial  \omega_2}&
%%		\dfrac{\partial  a_3}{\partial  \omega_3}&
%%		\dfrac{\partial  a_3}{\partial  \omega_4}
%%		\\[1em]
%%		\dfrac{\partial  a_4}{\partial  \omega_1}&
%%		\dfrac{\partial  a_4}{\partial  \omega_2}&
%%		\dfrac{\partial  a_4}{\partial  \omega_3}&
%%		\dfrac{\partial  a_4}{\partial  \omega_4}
%%		\\[1em]
%%		\dfrac{\partial  a_5}{\partial  \omega_1}&
%%		\dfrac{\partial  a_5}{\partial  \omega_2}&
%%		\dfrac{\partial  a_5}{\partial  \omega_3}&
%%		\dfrac{\partial  a_5}{\partial  \omega_4}
%%		\\[1em]
%%		\dfrac{\partial  a_6}{\partial  \omega_1}&
%%		\dfrac{\partial  a_6}{\partial  \omega_2}&
%%		\dfrac{\partial  a_6}{\partial  \omega_3}&
%%		\dfrac{\partial  a_6}{\partial  \omega_4}
%%		\\[1em]
%%	\end{bmatrix}
%\end{equation}
%ماتریس‌های $\boldsymbol A$ و $\boldsymbol B$ مطابق روابط
%(\ref{A_compact})
%تا
%(\ref{B_MIMO})
%محاسبه می‌شوند.
%\begin{equation}\label{A_compact}
%	\boldsymbol A = \begin{bmatrix}
%		\dfrac{\partial \boldsymbol{ f}}{\partial  x_1} &
%		\dfrac{\partial \boldsymbol{ f}}{\partial  x_2} &
%		\dfrac{\partial \boldsymbol{ f}}{\partial  x_3} &
%		\dfrac{\partial \boldsymbol{ f}}{\partial  x_4} &
%		\dfrac{\partial \boldsymbol{ f}}{\partial  x_5} &
%		\dfrac{\partial \boldsymbol{ f}}{\partial  x_6} 
%	\end{bmatrix}
%\end{equation}
%\begin{equation}
%	 \dfrac{\partial \boldsymbol{ f}}{\partial  x_1} = 
%	 \begin{bmatrix}
%	 	x_5 \,\mathrm{cos}\left(x_1 \right)\,\mathrm{tan}\left(x_2 \right)-x_6 \,\mathrm{sin}\left(x_1 \right)\,\mathrm{tan}\left(x_2 \right)\\
%	 	-x_6 \,\mathrm{cos}\left(x_1 \right)-x_5 \,\mathrm{sin}\left(x_1 \right)\\[0.5em]
%	 	\dfrac{x_5 \,\mathrm{cos}\left(x_1 \right)-x_6 \,\mathrm{sin}\left(x_1 \right)}{\mathrm{cos}\left(x_2 \right)}\\
%	 	A_1 \,\mathrm{cos}\left(x_1 \right)\,\mathrm{cos}\left(x_2 \right)\\
%	 	0\\
%	 	0
%	 \end{bmatrix}
%\end{equation}
%\begin{equation}
%	\dfrac{\partial \boldsymbol{ f}}{\partial  x_2} = 
%	\begin{bmatrix}
%	\dfrac{x_6 \,\mathrm{cos}\left(x_1 \right)}{{\mathrm{cos}\left(x_2 \right)}^2 }+\dfrac{x_5 \,\mathrm{sin}\left(x_1 \right)}{{\mathrm{cos}\left(x_2 \right)}^2 }\\
%	0\\[0.5em]
%	\dfrac{\mathrm{tan}\left(x_2 \right)\,{\left(x_6 \,\mathrm{cos}\left(x_1 \right)+x_5 \,\mathrm{sin}\left(x_1 \right)\right)}}{\mathrm{cos}\left(x_2 \right)}\\
%	-A_2 \,\mathrm{sin}\left(x_1 \right)\,\mathrm{sin}\left(x_2 \right)\\
%	B_1 \,\mathrm{cos}\left(x_2 \right)\\
%	0
%	\end{bmatrix}
%\end{equation}
%\begin{equation}
%	\dfrac{\partial \boldsymbol{ f}}{\partial  x_3} = 
%	\begin{bmatrix}
%		0\\0\\0\\0\\0\\0
%	\end{bmatrix}
%\end{equation}
%\begin{equation}
%	\dfrac{\partial \boldsymbol{ f}}{\partial  x_4} = 
%	\begin{bmatrix}
%	1\\
%	0\\
%	0\\
%	0\\
%	B_2 \,x_6 +B_4 \,{\left(\omega_1 -\omega_2 +\omega_3 -\omega_4 \right)}\\
%	C_1 \,x_5 
%	\end{bmatrix}
%\end{equation}
%\begin{equation}
%	\dfrac{\partial \boldsymbol{ f}}{\partial  x_5} = 
%	\begin{bmatrix}
%	\mathrm{sin}\left(x_1 \right)\,\mathrm{tan}\left(x_2 \right)\\
%	\mathrm{cos}\left(x_1 \right)\\
%	\dfrac{\mathrm{sin}\left(x_1 \right)}{\mathrm{cos}\left(x_2 \right)}\\
%	A_2 \,x_6 +A_4 \,{\left(\omega_1 -\omega_2 +\omega_3 -\omega_4 \right)}\\
%	0\\
%	C_1 \,x_4 
%	\end{bmatrix}
%\end{equation}
%\begin{equation}
%	\dfrac{\partial \boldsymbol{ f}}{\partial  x_6} = 
%	\begin{bmatrix}
%	\mathrm{cos}\left(x_1 \right)\,\mathrm{tan}\left(x_2 \right)\\
%	-\mathrm{sin}\left(x_1 \right)\\
%	\dfrac{\mathrm{cos}\left(x_1 \right)}{\mathrm{cos}\left(x_2 \right)}\\
%	0\\
%	B_2 \,x_4 \\
%	0
%	\end{bmatrix}
%\end{equation}
%
%
%\begin{equation}\label{B_MIMO}
%	\boldsymbol B = \begin{bmatrix}
%			0 & 0 & 0 & 0\\
%			0 & 0 & 0 & 0\\
%			0 & 0 & 0 & 0\\
%			A_4 \,x_5  & 2\,A_3 \,\omega_2 -A_4 \,x_5  & A_4 \,x_5  & -2\,A_3 \,\omega_4 -A_4 \,x_5 \\
%			2\,B_3 \,\omega_1 +B_4 \,x_4  & -B_4 \,x_4  & B_4 \,x_4 -2\,B_3 \,\omega_3  & -B_4 \,x_4 \\
%			2\,C_2 \,\omega_1  & -2\,C_2 \,\omega_2  & 2\,C_2 \,\omega_3  & -2\,C_2 \,\omega_4 
%	\end{bmatrix}
%\end{equation}
