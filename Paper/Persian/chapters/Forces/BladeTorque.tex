\subsection{گشتاورهای ناشی از آيرودينامیک پره‌ها}\label{sec:aeromoment}
آیرودینامیک پره‌ها باعث ایجاد نیروی برآ و درنتیجه گشتاورهای رول و پیچ ناشی از اختلاف نیروی 
برآ می‌شود. با استفاده از تفاضل نیروی برآی پره‌ها دو گشتاور رول و پیچ ایجاد می‌شود. با توجه به تئوری مومنتوم، نیروی برآی هر پره 
% farsi momentum theory
$(T_i)$
از رابطه‌ی زیر حاصل می‌شود
\cite{Sharifi}:
\begin{equation}\label{trust}
	T_i = b\omega_i^2
\end{equation}
در رابطه
(\ref{trust})
$b$
و 
$\omega_i$
به ترتیب بیانگر فاکتور نیروی برآ و سرعت زاویه‌ای هر پره است؛ بنابراین مطابق شکل 
\ref{QuadAssum}
گشتاور رول حول محور
$X^B$
دستگاه مختصات بدنی از رابطه زیر حاصل می‌شود.
\begin{equation}\label{roll}
	m_X^B = d_{cg}(T_2-T_4) = d_{cg}b(\omega_2^2-\omega_4^2)
\end{equation}
در رابطه 
(\ref{roll})
عبارت 
$d_{cg}$
بیانگر فاصله مرکز هر پره از مرکز جرم چهارپره در راستای محور
$X^B$
دستگاه مختصات بدنی است. همچنین گشتاور پیچ حول محور 
$Y^B$
دستگاه مختصات بدنی با توجه به شكل
\ref{QuadAssum}
از رابطه زیر حاصل می‌شود:
\begin{equation}\label{pitch}
	m_Y^B = d_{cg}(T_1-T_3) = d_{cg}b(\omega_1^2-\omega_3^2)
\end{equation}
گشتاور یاو آیرودینامیكی از اختلاف گشتاور ناشی از پسای پره‌ها ایجاد می‌شود؛ بنابراین، جهت این 
گشتاور همواره در جهت مخالف چرخش پره‌ها است. بنابراین، گشتاور یاو حول محور
$Z^B$
دستگاه مختصات بدنی با توجه به شكل
\ref{QuadAssum},
مطابق رابطه زیر حاصل می‌شود:
\begin{equation}\label{yaw}
	m_Z^B = d(\omega_1^2-\omega_2^2+\omega_3^2-\omega_4^2)
\end{equation}
رابطه 
(\ref{yaw})
عبارت 
$d$
بیانگر فاکتور گشتاور پسای پره‌ها است. در نتیجه با توجه به معادلات
(\ref{roll}),
(\ref{pitch})
و
(\ref{yaw})
بردار گشتاورهای خارجی ناشی از آیرودینامیک پره‌ها در دستگاه مختصات بدنی به‌صورت زیر حاصل می‌شود:
\begin{equation}\label{finaltorque}
	\left[m_A\right]^B = \begin{bmatrix}
		m_X^B\\m_Y^B\\m_Z^B
	\end{bmatrix}
 =  \begin{bmatrix}
 	d_{cg}b(\omega_2^2-\omega_4^2)\\
 	d_{cg}b(\omega_1^2-\omega_3^2)\\
 	d(\omega_1^2-\omega_2^2+\omega_3^2-\omega_4^2)
 \end{bmatrix}
\end{equation}