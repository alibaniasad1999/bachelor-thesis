\chapter{نتیجه‌گیری}
در این پایان‌نامه، یک کنترل‌کننده مبتنی بر بازی دیفرانسیلی به منظور کنترل وضعیت یک استند آزمایشگاهی سه درجه آزادی چهارپره طراحی و پیاده‌سازی شد. 
به این منظور، ابتدا مدل‌سازی سه درجه آزادی وضعیت وسیله با در نظر گرفتن گشتاورهای خارجی انجام شد. پس از پیاده‌سازی مدل ریاضی، اصلاح پارامتر انجام شد. سپس، صحت عملکرد شبیه‌سازی استند سه درجه آزادی چهارپره با نتایج تست آزمایشگاهی ارزیابی شد.
	 از کنترل‌کننده بهینه \lr{LQR} برای کنترل آن استفاده شد. از آنجا که عملكرد سامانه در حضور اغتشاش مطلوب نبود، کنترل‌کننده‌های \lr{LQDG} و سپس، \lr{LQIDG} طراحی، شبیه‌سازی  و روی سامانه پیاده‌سازی شد. نتایج پیاده‌سازی نشان داد کنترل‌کننده \lr{LQR} درحضور اغتشاش با نوسان زیادی همراه است و کنترل‌کننده \lr{LQDG} در حضور اغتشاش قادر به صفر کردن خطای حالت ماندگار و نوسان نخواهد بود. بنابراین، برای حل این مشكل، کنترل‌کننده \lr{LQIDG} استفاده شد. نتایج شبیه‌سازی نشان داد که این کنترل‌کننده، علاوه بر صفرکردن خطای حالت ماندگار، فراجهش کمتری نسبت به کنترل‌کننده \lr{LQR} دارد. در نهایت، بهبودهای حاصل‌شده در عملكرد کنترل‌کننده، با پیاده‌سازی آن بر روی سامانه، تایید شد.
	 کدهای استفاده شده در این پایان‌نامه به‌صورت منبع باز\LTRfootnote{Open Source} 
	 در دسترس است\LTRfootnote{https://github.com/alibaniasad1999/bachelor-thesis}.
\newpage



 \section{نوآوری‌های پایان‌نامه}
 نوآوری‌های این پایان‌نامه شامل موارد زیر است:
 \begin{itemize}
 	\item پیاده‌سازی کنترل‌کننده \lr{LQIDG} روی سامانه کنترل وضعیت سه درجه آزادی چهارپره.
 	\item مقایسه نتایج شبیه‌سازی و پیاده‌سازی سه کنترل‌کننده بهینه در حضور اغتشاش مدل‌سازی
 \end{itemize}

\section{پیشنهادها برای ادامه کار}
پیشنهادهایی که برای ادامه این کار وجود دارد، شامل موارد زیر است:
 \begin{itemize}
	\item حذف تاخیر حسگر به کمک فیلتر کالمن
	\item بررسی عملكرد کنترل‌کننده \lr{LQIDG} روی سامانه‌های دیگر و در شرایط اغتشاشی مختلف
\end{itemize}