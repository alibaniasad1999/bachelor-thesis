

\documentclass[12 pt]{article}
% Template-specific packages

\usepackage{graphicx} % Required for including images


\usepackage{fullpage,enumitem,amsmath,amssymb,graphicx}
\usepackage{pdfpages}
\usepackage{amsmath, amsthm}
\usepackage{fancyhdr}
\pagestyle{fancy}
\fancyhead[R]{\rightmark}
\fancyhead[L]{Ali BaniAsad 96108378}
\setlength{\headheight}{10pt}
\setlength{\headsep}{0.2in}
\usepackage{titling}
\usepackage{float}
%----------------------------------------------------------------------------------------
%	ASSIGNMENT INFORMATION
%----------------------------------------------------------------------------------------



%----------------------------------------------------------------------------------------
\usepackage{xepersian}
\settextfont{Yas}
\usepackage{graphicx}
\title{تعریف پروژه کارشناسی}
\author{علی بنی اسد 96108378}

\begin{document}
	\maketitle
	\section*{کنترل وضعیت استند چهارپره
		\LTRfootnote{\lr{Quadcopter}}
		 سه درجه آزادی 
		با استفاده از روش کنترل کننده خطی مبتنی بر بازی دیفرانسیلی
(\lr{LQDG\LTRfootnote{\lr{Linear Quadratic Differential Game}}})	}





چهارپره یکی از انواع پرنده‌های بدون سرنشین عمودپرواز است. از ویژگی‌های بارز این پرنده‌ها می‌توان به کوپلینگ بین کانال‌های رول\LTRfootnote{Roll}، پیچ\LTRfootnote{Pitch} و یاو\LTRfootnote{Yaw} اشاره کرد. از این‌رو، با استفاده از یک مکانیزم ‌سه‌درجه‌آزادی آزمایشگاهی می‌توان نحوه کنترل همزمان زوایای رول، پیچ و یاو چهارپره را بررسی کرد. در این میان، نرم‌افزار‌ سیمولینک امکان ارتباط با حسگرها، کنترل‌کننده و فرمان به موتور را در یک محیط یکپارچه و به ‌صورت بلوک‌های گرافیکی به کاربر می‌دهد که سرعت درک و خطا‌یابی سیستم را دوچندان می‌کند.



هدف از این پروژه کنترل استند چهارپره سه درجه‌آزادی با استفاده از روش LQDG است. در فرضیات پروژه اغتشاش به صورت یک بازیکن در نظر گرفته شده‌است که با سیستم همکاری ندارد. هر یک از بازیکنان(سیستم و اغتشاش) سعی در کم کردن تابع هزینه خود با فرض بدترین حرکت طرف مقابل دارند. اساس این روش بر بهینه سازی تابع هزینه سیستم‌های دیفرانسیلی مبنتی بر تعادل نش 
\LTRfootnote{Nash Equilibrium}
است. در انتها عملکرد این کنترل کننده با تعقیب زوایای مطلوب پله، موج مربعی و سینوسی بررسی و با نتایج حال از کنترل‌کننده  تناسبی - انتگرالی -
مشتقی\LTRfootnote{PID(Proportional–Integral–Derivative)} مقایسه خواهد‌شد.
\end{document}

