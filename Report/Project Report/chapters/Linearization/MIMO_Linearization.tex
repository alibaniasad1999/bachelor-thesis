\subsection{خطی‌سازی به فرم چند ورودی چند خروجی}\label{lin_MIMO}
در این قسمت با توجه به فضای حالت بدست آمده، چهارپره حول نقطه کار خطی‌سازی می‌شود.
\begin{equation*}
	\boldsymbol a = \begin{bmatrix}
		x_4 + x_5\sin(x_1)\tan(x_2) + x_6\cos(x_1)\tan(x_2)\\
		x_5\cos(x_1)- x_6\sin(x_1)\\
		(x_5\sin(x_1) + x_6\cos(x_1))\sec(x_2)\\
		A_1\cos(x_2)\sin(x_1) + 
		A_2x_5x_6 + A_3\left(\omega_2^2-\omega_4^2\right)+
		A_4x_5\left(\omega_1-\omega_2+\omega_3-\omega_4\right)\\
		B_1\sin(x_2) + 
		B_2x_4x_6 + B_3\left(\omega_1^2-\omega_3^2\right)+
		B_4x_4\left(\omega_1-\omega_2+\omega_3-\omega_4\right)\\
		C_1x_4x_5 + 
		C_2\left(\omega_1^2-\omega_2^2+\omega_3^2-\omega_4^2\right)
	\end{bmatrix}
\end{equation*} 
\begin{equation}
	\vec{x} = \begin{bmatrix} % bold or vec???????????/
		\phi& \theta & \psi & p& q& r
	\end{bmatrix}^T
\end{equation}
\begin{equation}
	\vec{\omega} = \begin{bmatrix}
		\omega_1&\omega_2&\omega_3&\omega_4
	\end{bmatrix}^T
\end{equation}
برای خطی سازی از بسط تیلور استفاده شده ‌است.
\begin{equation}
	\delta \dot{\vec{x}} = \dfrac{\partial \vec a}{\partial \vec x}\delta \vec x + \dfrac{\partial \vec a}{\partial \vec\omega}\delta \vec \omega 
\end{equation}
\begin{equation}
	\dot{\vec{x}} =
	\begin{bmatrix}
		\delta \dot x_1&
		\delta \dot x_2&
		\delta \dot x_3&
		\delta \dot x_4&
		\delta \dot x_5&
		\delta \dot x_6
	\end{bmatrix}^T
\end{equation}

\begin{equation}
	\boldsymbol A = \dfrac{\partial \vec a}{\partial \vec x} =
	\begin{bmatrix}
\dfrac{\partial  a_1}{\partial  x_1}&
\dfrac{\partial  a_1}{\partial  x_2}&
\dfrac{\partial  a_1}{\partial  x_3}&
\dfrac{\partial  a_1}{\partial  x_4}&
\dfrac{\partial  a_1}{\partial  x_5}&
\dfrac{\partial  a_1}{\partial  x_6}
\\[1em]
\dfrac{\partial  a_2}{\partial  x_1}&
\dfrac{\partial  a_2}{\partial  x_2}&
\dfrac{\partial  a_2}{\partial  x_3}&
\dfrac{\partial  a_2}{\partial  x_4}&
\dfrac{\partial  a_2}{\partial  x_5}&
\dfrac{\partial  a_2}{\partial  x_6}
\\[1em]
\dfrac{\partial  a_3}{\partial  x_1}&
\dfrac{\partial  a_3}{\partial  x_2}&
\dfrac{\partial  a_3}{\partial  x_3}&
\dfrac{\partial  a_3}{\partial  x_4}&
\dfrac{\partial  a_3}{\partial  x_5}&
\dfrac{\partial  a_3}{\partial  x_6}
\\[1em]
\dfrac{\partial  a_4}{\partial  x_1}&
\dfrac{\partial  a_4}{\partial  x_2}&
\dfrac{\partial  a_4}{\partial  x_3}&
\dfrac{\partial  a_4}{\partial  x_4}&
\dfrac{\partial  a_4}{\partial  x_5}&
\dfrac{\partial  a_4}{\partial  x_6}
\\[1em]
\dfrac{\partial  a_5}{\partial  x_1}&
\dfrac{\partial  a_5}{\partial  x_2}&
\dfrac{\partial  a_5}{\partial  x_3}&
\dfrac{\partial  a_5}{\partial  x_4}&
\dfrac{\partial  a_5}{\partial  x_5}&
\dfrac{\partial  a_5}{\partial  x_6}
\\[1em]
\dfrac{\partial  a_6}{\partial  x_1}&
\dfrac{\partial  a_6}{\partial  x_2}&
\dfrac{\partial  a_6}{\partial  x_3}&
\dfrac{\partial  a_6}{\partial  x_4}&
\dfrac{\partial  a_6}{\partial  x_5}&
\dfrac{\partial  a_6}{\partial  x_6}
	\end{bmatrix}
\end{equation}
\begin{equation}
	\boldsymbol B = \dfrac{\partial \vec a}{\partial \vec \omega} = 
	\begin{bmatrix}
		\dfrac{\partial  a_1}{\partial  \omega_1}&
		\dfrac{\partial  a_1}{\partial  \omega_2}&
		\dfrac{\partial  a_1}{\partial  \omega_3}&
		\dfrac{\partial  a_1}{\partial  \omega_4}
		\\[1em]
		\dfrac{\partial  a_2}{\partial  \omega_1}&
		\dfrac{\partial  a_2}{\partial  \omega_2}&
		\dfrac{\partial  a_2}{\partial  \omega_3}&
		\dfrac{\partial  a_2}{\partial  \omega_4}
		\\[1em]
		\dfrac{\partial  a_3}{\partial  \omega_1}&
		\dfrac{\partial  a_3}{\partial  \omega_2}&
		\dfrac{\partial  a_3}{\partial  \omega_3}&
		\dfrac{\partial  a_3}{\partial  \omega_4}
		\\[1em]
		\dfrac{\partial  a_4}{\partial  \omega_1}&
		\dfrac{\partial  a_4}{\partial  \omega_2}&
		\dfrac{\partial  a_4}{\partial  \omega_3}&
		\dfrac{\partial  a_4}{\partial  \omega_4}
		\\[1em]
		\dfrac{\partial  a_5}{\partial  \omega_1}&
		\dfrac{\partial  a_5}{\partial  \omega_2}&
		\dfrac{\partial  a_5}{\partial  \omega_3}&
		\dfrac{\partial  a_5}{\partial  \omega_4}
		\\[1em]
		\dfrac{\partial  a_6}{\partial  \omega_1}&
		\dfrac{\partial  a_6}{\partial  \omega_2}&
		\dfrac{\partial  a_6}{\partial  \omega_3}&
		\dfrac{\partial  a_6}{\partial  \omega_4}
		\\[1em]
	\end{bmatrix}
\end{equation}
به علت حجم بالای معادلات رابطه خطی‌سازی شده چهارپره در گزارش آورده نشده‌است اما در شبیه سازی به طور کامل لحاظ شده‌است.
% benevisam ya na?????????//?
