\chapter{بازی دیفرانسیلی}
در این قسمت به خلاصه‌ای از بازی دیفرانسیلی پرداخته شده است. تمامی توضیحات و روابط  از منبع 
 \cite{article1}
آورده شده است. در این فصل حالت حلقه‌باز\LTRfootnote{Opne Loop}
 و حالت همراه با بازخورد\LTRfootnote{Feedback} بررسی می‌شود.
 % یا شده است؟
 این پروژه حالت دو بازیکن را بررسی می‌کند. در این مسئله فرض شده  که تابع هزینه برای هر بازیکن به فرم مربعی است. 	هدف اصلی پروژه کم کردن تابع هزینه برای بازیکنان است. تابع هزینه به فرم رابطه \ref{cost}نوشته می‌شود.

 \begin{equation}\label{cost}
 	J_i(u_1, u_2) = \int_{0}^{T}\left( \boldsymbol{x} ^T(t) \boldsymbol{Q_i} \boldsymbol{x}(t)+
 	 \boldsymbol{u_i} ^T(t) \boldsymbol{Q_{ii}} \boldsymbol{u_i}(t)+
 	 \boldsymbol{u_j} ^T(t)\boldsymbol{ Q_{ij} u_j}(t)
 	\right)dt+
 	\boldsymbol{ x} ^T(T)\boldsymbol{ H_i}\boldsymbol{ x}(T) 
  \end{equation}
در اینجا ماتریس‌های 
$\boldsymbol{Q_i}$ ، $\boldsymbol{R_{ii}}$
و
$\boldsymbol{H}$
متقارن فرض شده‌اند و ماتریس 
$\boldsymbol{R_{ii}}$
به صورت مثبت معین ($\boldsymbol{R_{ii}}>0$)
فرض شده‌است.
دینامیک سیستم تحت تاثیر هر دو بازیکن قرار می‌گیرد. در اینحا دینامک سیستم به فرم رابطه \ref{system_dynamic} در نظر گرفته شده ‌است.
\begin{equation}\label{system_dynamic}
	\boldsymbol{\dot x}(t) = \boldsymbol{Ax}(t) + \boldsymbol{B_1u_1}(t) + \boldsymbol{B_2u_2}(t), \quad x(0) = x_0
\end{equation}
در رابطه 
\ref{system_dynamic}
$\boldsymbol{u_1}$
برابر با تلاش کنترلی بهینه بازیکن اول است. در اینحا ممکن است تلاش کنترلی بازیکن اول موجب دور شدن بازیکن دوم از هدف شود و یا برعکس.  این پروژه حالت همکاری دو بازیکن را بررسی نمی‌کند و دو بازیکن در تلاش برای کم کردن تابع هزینه خود و زیاد کردن تابع هزینه بازیکن مقابل هستند.
