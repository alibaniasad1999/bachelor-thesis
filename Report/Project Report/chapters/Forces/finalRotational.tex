\subsection{ استخراج معادله نهایی دينامیک دورانی}
در این بخش، گشتاورهای خارجی چهارپره و تكانه زاوی‌های کلی پره‌ها در معادله دیفرانسیل اویلر 
جایگذاری شده و شكل نهایی معادله دیفرانسیل استند چهارپره حاصل می‌شود. با جایگذاری مقدار 
گشتاورهای اعمالی به چهارپره از معادله
\ref{finalm}
در معادله 
\ref{torque}
رابطه موردنیاز برای مدل‌سازی
دینامیک دورانی استند بهصورت معادله زیر حاصل می‌شود:
\begin{equation}\label{demifinalrotation}
	\begin{split}
		\left[\dfrac{d\boldsymbol\omega^{BI}}{dt}\right]^B = 
		\left(\left[\boldsymbol J\right]^B\right)^{-1}\Bigg(
		-&\left[\boldsymbol\Omega^{BI}\right]\times\left(
		\left[\boldsymbol J\right]^B\left[\boldsymbol\omega^{BI}\right]^B
		+\left[\boldsymbol I_R\right]^B\right) + \\
		&\left[\boldsymbol m_A\right]^B+\left[\boldsymbol s_{DC}\right]^B
		\bigg(-\left[\boldsymbol G\right]^B
		-\left[\boldsymbol T\right]^B+ \\
		&m_{tot}\Bigl\{2
		\left[\boldsymbol \Omega^{BI}\right]^B
		\left[\boldsymbol \Omega^{BI}\right]^B
		\left[\boldsymbol s_{cd}\right]^B+
		\left[\dfrac{d\boldsymbol\Omega^{BI}}{dt}\right]^B
		\left[\boldsymbol s_{cd}\right]^B
		\Bigr\}\bigg)\Bigg)
	\end{split}
\end{equation}
در رابطه
\ref{demifinalrotation}،
عبارت
$\left[\dfrac{d\boldsymbol\Omega^{BI}}{dt}\right]^B$
بیانگر ماتریس پادمتقارن بردار مشتق سرعت‌ زاویه‌ای بدنی
$\left[\dfrac{d\boldsymbol\omega^{BI}}{dt}\right]^B$
است. جمله آخر در معادله فوق را می‌توان به صورت زیر بازنویسی کرد:
\begin{equation}\label{sc}
	m_{tot}\left[\boldsymbol s_{DC}\right]^R
	\left[\dfrac{d\boldsymbol\Omega^{BI}}{dt}\right]^B\left[\boldsymbol s_{CD}\right]^R = 
	m_{tot}\left[\boldsymbol s_{DC}\right]^R\left[\boldsymbol s_{DC}\right]^R
	\left[\dot{\boldsymbol\omega}^{BI}\right]^B
\end{equation}
با جایگذاری معادله
\ref{sc}
در معادله
\ref{demifinalrotation}
معادله زیر حاصل می‌شود:
\begin{equation}\label{longfinal}
	\begin{split}
		\left[\dfrac{d\boldsymbol\omega^{BI}}{dt}\right]^B  & \left(
		I - m_{tot} \left(\left[\boldsymbol J\right]^B\right)^{-1}
		\left[\boldsymbol s_{DC}\right]^B
		\left[\boldsymbol s_{DC}\right]^B\right) = \\
		\left(\left[\boldsymbol J\right]^B\right)^{-1}\Bigg(
		-&\left[\boldsymbol\Omega^{BI}\right]\times\left(
		\left[\boldsymbol J\right]^B\left[\boldsymbol\omega^{BI}\right]^B
		+\left[\boldsymbol I_R\right]^B\right) + \\
		&\left[\boldsymbol m_A\right]^B+\left[\boldsymbol s_{DC}\right]^B
		\bigg(-\left[\boldsymbol F_g\right]^B
		-\left[\boldsymbol F_T\right]^B+ \\
		&m_{tot}\Bigl\{2
		\left[\boldsymbol\Omega^{BI}\right]^B
		\left[\boldsymbol\Omega^{BI}\right]^B
		\left[\boldsymbol s_{cd}\right]^B+
		\left[\dfrac{d\boldsymbol\Omega^{BI}}{dt}\right]^B
		\left[\boldsymbol s_{cd}\right]^B
		\Bigr\}\bigg)\Bigg)		
	\end{split}
\end{equation}
با ساده‌سازی رابطه \ref{longfinal}  بردار سرعت زاویه‌ای استند به صورت زیر حاصل می‌شود:
\begin{equation}\label{finaltorquefinal}
	\begin{split}
		\left[\dfrac{d\boldsymbol\omega^{BI}}{dt}\right]^B &= a^{-1}b\\
		a &= I - m_{tot}\left(\left[\boldsymbol J\right]^B\right)^{-1}
		\left[\boldsymbol s_{DC}\right]^B
		\left[\boldsymbol s_{DC}\right]^B\\
		b = \left(\left[\boldsymbol J\right]^B\right)^{-1}\Bigg(
		-&\left[\boldsymbol\Omega^{BI}\right]\times\left(
		\left[\boldsymbol J\right]^B\left[\boldsymbol\omega^{BI}\right]^B
		+\left[\boldsymbol I_R\right]^B\right) + \\
		&\left[\boldsymbol m_A\right]^B+\left[\boldsymbol s_{DC}\right]^B
		\bigg(-\left[\boldsymbol F_g\right]^B
		-\left[\boldsymbol F_T\right]^B+ \\
		&m_{tot}\Bigl\{2
		\left[\boldsymbol\Omega^{BI}\right]^B
		\left[\boldsymbol\Omega^{BI}\right]^B
		\left[\boldsymbol s_{cd}\right]^B+
		\left[\dfrac{d\boldsymbol\Omega^{BI}}{dt}\right]^B
		\left[\boldsymbol s_{cd}\right]^B
		\Bigr\}\bigg)\Bigg)		
	\end{split}
\end{equation}
با جایگذاری معادلات
 \ref{IR}،
 \ref{IRomega} و
 \ref{omega_d}
 معادله مربوط به تکانه کلی پره‌ها، 
 معادله \ref{trust}
مربوط به برآی پره و معادله 
\ref{finaltorque}
در معادله 
\ref{finaltorquefinal}،
مؤلفه‌های بردار مشتق سرعت زاویه‌ای چهارپره به صورت زیر حاصل می‌شود:
\begin{align}
	\begin{split}
		\dot{p} =& \dfrac{h_{cg}gm_{dot}\cos(\theta)\sin(\phi)
			+\left(J_{22} - J_{33} +2m_{tot}h_{ch}^2\right)qr
		}
		{m_{tot}h_{cg}^2 + J_{11}} \\
		&+\dfrac{bd_{cg}\left(\omega_2^2-\omega_4^2\right) + qJ_R(\omega_1-\omega_2+\omega_3-\omega_4)}
		{m_{tot}h_{cg}^2 + J_{11}}
	\end{split}\\[1em]
		\begin{split}
		\dot{q} =& \dfrac{h_{cg}gm_{dot}\sin(\theta)
			+\left(J_{33} - J_{11} +2m_{tot}h_{ch}^2\right)pr
		}
		{m_{tot}h_{cg}^2 + J_{11}} \\
		&+\dfrac{bd_{cg}\left(\omega_1^2-\omega_3^2\right) - pJ_R(\omega_1-\omega_2+\omega_3-\omega_4)}
		{m_{tot}h_{cg}^2 + J_{11}}
	\end{split}\\[1em]
	\begin{split}
		\dot{r} =& \dfrac{pq(J_{11}-J_{22})
		+ d(\omega_1^2-\omega_2^2+\omega_3^2-\omega_4^2)
	}{J_{33}}
	\end{split}
\end{align}
به منظور انتشار وضعیت دورانی چهارپره، از روش انتشار اویلر استفاده می‌شود. در این‌صورت
\cite{zipfel2000modeling}
:
$$\dot\phi = p + q\sin(\phi)\cos(\theta) +‌
r\cos(\phi)\tan(\theta)
$$
$$
\dot \theta = q\cos(\phi) - r\sin(\phi)
$$
$$
\dot\psi = (q\sin(phi)) + r\cos(\phi))\sec(\theta) 
$$

