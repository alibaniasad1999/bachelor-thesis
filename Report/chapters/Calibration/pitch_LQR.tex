\section{پیاده‌سازی کنترل کننده \lr{LQR} بر رویه کانال پیچ استند}
در بخش
\ref{roll_lqr_section_simulation}
شبیه‌سازی تک کانال استند چهارپره در حضور کنترل‌کننده \lr{LQR} انجام شد. در این بخش به پیاده‌سازی کنترل‌کننده \lr{LQR} بر رویه کانال پیچ استند سه درجه آزادی پرداخته می‌شود. در پیاده‌سازی از ضرایب وزنی بهینه به دست آمده در قسمت شبیه‌سازی استفاده شده‌است.
\begin{figure}
	\includegraphics[width=.48\linewidth]{../Figures/Calibration/LQR/Pitch/lqr_pitch.png}
	\centering
	\caption{عملكرد \lr{LQR} در کنترل زاويه پیچ (تعقیب ورودی صفر)}
\end{figure}
%
%\begin{figure}[H]
%	[width=12cm]
%	\centering
%	\begin{subfigure}
%		\centering
%		\includegraphics[width=12cm]{../Figures/Calibration/LQR/Pitch/lqr_pitch_Omega_1.png}
%		\caption{موتور شماره یک}
%	\end{subfigure}
%	\begin{subfigure}
%		\centering
%		\includegraphics[width=12cm]{../Figures/Calibration/LQR/Pitch/lqr_pitch_Omega_3.png}
%		\caption{موتور شماره سه}
%	\end{subfigure}
%	\caption{‫‪فرمان کنترل‌کننده موتور سه و چهار در کنترل زاویه رول و پیچ (تعقیب ورودی صفر)}
%\end{figure}
\begin{figure}[H]
	\centering
	\subfigure[موتور شماره یک]{
		\centering
		\includegraphics[width=.45\linewidth]{../Figures/Calibration/LQR/Pitch/lqr_pitch_Omega_1.png}
	}
	\subfigure[موتور شماره سه]{
		\centering
		\includegraphics[width=.45\linewidth]{../Figures/Calibration/LQR/Pitch/lqr_pitch_Omega_3.png}
	}
	\caption{‫‪فرمان کنترلی موتورهای دو چهار در کنترل زاویه پیچ (تعقیب ورودی صفر)}
\end{figure}

%بر اساس خروجی شبیه‌سازی (شکل
%\ref{lqr_roll_fig})
%،کانال رول در حضور کنترل‌کننده \lr{LQR} در حدود پنج ثانیه به تعادل می‌رسد اما دارای خطای ماندگار است. 