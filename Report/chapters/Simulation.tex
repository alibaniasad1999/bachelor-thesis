\chapter{نتایج شبیه‌سازی}
در حل معادله کوپل ریکاتی برای سادگی از سیستم خطی‌سازی شده در شرایط تعادل حول نقطه صفر استفاده‌شده‌است. اما در شبیه‌سازی سیستم، از سیستم خطی شده حول نقطه کار استفاده‌شده‌است. این کار دقت حل را کاهش می‌دهد اما سرعت سیستم را به شدت بالا می‌برد از طرفی بر اساس نتایج شبیه این روش عمکرد خوبی دارد و مشکلی برای سیستم ایجاد نمی‌کند. شبیه‌سازی در شرایطی تعادش نش دو بازیکن آورده‌شده‌است و هر دو بازیکن در سیستم تغیرات ایجاد می‌کنند اما فقط تلاش کنترلی بازیکن اول آورده شده‌است.
\section{شبیه‌ساز حلقه‌باز}
در شبیه‌سازی هفت حالا مختلف آزمایش شده‌است که در اینجا نتایح و خروجی شبیه‌‌سازی آورده‌شده‌است.
\begin{figure}[H]
	\includegraphics[width=12cm]{../Figure/LQDG/OpenLoopLQDGroll.png}
	\centering
	\caption{شبیه‌‌سازی چهارپره حلقه باز در حالت انحراف رول یک رادیان}
\end{figure}
\begin{figure}[H]
	\includegraphics[width=12cm]{../Figure/LQDG/OpenLoopLQDGrollcontrol.png}
	\centering
	\caption{تلاش کنترلی بازیکن اول چهارپره حلقه باز در حالت انحراف رول یک رادیان}
\end{figure}
	\begin{figure}[H]
		\includegraphics[width=12cm]{../Figure/LQDG/OpenLoopLQDGpitch.png}
		\centering
		\caption{شبیه‌‌سازی چهارپره حلقه باز در حالت انحراف پیچ یک رادیان}
	\end{figure}
	\begin{figure}[H]
		\includegraphics[width=12cm]{../Figure/LQDG/OpenLoopLQDGpitchcontrol.png}
		\centering
		\caption{تلاش کنترلی بازیکن اول چهارپره حلقه باز در حالت انحراف پیچ یک رادیان}
\end{figure}
	\begin{figure}[H]
	\includegraphics[width=12cm]{../Figure/LQDG/OpenLoopLQDGpsi.png}
	\centering
	\caption{شبیه‌‌سازی چهارپره حلقه باز در حالت انحراف یاو یک رادیان}
\end{figure}
\begin{figure}[H]
	\includegraphics[width=12cm]{../Figure/LQDG/OpenLoopLQDGpsicontrol.png}
	\centering
	\caption{تلاش کنترلی بازیکن اول چهارپره حلقه باز در حالت انحراف یاو یک رادیان}
\end{figure}
	\begin{figure}[H]
	\includegraphics[width=12cm]{../Figure/LQDG/OpenLoopLQDGp.png}
	\centering
	\caption{شبیه‌‌سازی چهارپره حلقه باز در حالت انحراف سرعت زاویه‌ای p 
		 یک رادیان بر ثانیه}
\end{figure}
\begin{figure}[H]
	\includegraphics[width=12cm]{../Figure/LQDG/OpenLoopLQDGpcontrol.png}
	\centering
	\caption{تلاش کنترلی بازیکن اول چهارپره حلقه باز در حالت انحراف سذعت زاویه‌ای p 
		یک رادیان بر ثانیه}
\end{figure}
\begin{figure}[H]
\includegraphics[width=12cm]{../Figure/LQDG/OpenLoopLQDGq.png}
\centering
\caption{شبیه‌‌سازی چهارپره حلقه باز در حالت انحراف سرعت زاویه‌ای q 
	یک رادیان بر ثانیه}
\end{figure}
\begin{figure}[H]
\includegraphics[width=12cm]{../Figure/LQDG/OpenLoopLQDGqcontrol.png}
\centering
\caption{تلاش کنترلی بازیکن اول چهارپره حلقه باز در حالت انحراف سذعت زاویه‌ای q 
	یک رادیان بر ثانیه}
\end{figure}
\begin{figure}[H]
\includegraphics[width=12cm]{../Figure/LQDG/OpenLoopLQDGr.png}
\centering
\caption{شبیه‌‌سازی چهارپره حلقه باز در حالت انحراف سرعت زاویه‌ای r 
	یک رادیان بر ثانیه}
\end{figure}
\begin{figure}[H]
\includegraphics[width=12cm]{../Figure/LQDG/OpenLoopLQDGrcontrol.png}
\centering
\caption{تلاش کنترلی بازیکن اول چهارپره حلقه باز در حالت انحراف سذعت زاویه‌ای r 
	یک رادیان بر ثانیه}
\end{figure}
\begin{figure}[H]
	\includegraphics[width=12cm]{../Figure/LQDG/OpenLoopLQDGall.png}
	\centering
	\caption{شبیه‌‌سازی چهارپره حلقه باز در حالت انحراف  تمامی وضعیت به اندازه یک رادیان(رادیان بر ثانیه)}
\end{figure}
\begin{figure}[H]
	\includegraphics[width=12cm]{../Figure/LQDG/OpenLoopLQDGallcontrol.png}
	\centering
	\caption{تلاش کنترلی بازیکن اول چهارپره حلقه باز در حالت انحراف تمامی وضعیت به اندازه یک رادیان(رادیان بر ثانیه)  }
\end{figure}
