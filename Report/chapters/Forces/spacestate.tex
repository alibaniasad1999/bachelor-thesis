\section{استخراج فرم فضای حالت}\label{spacestate}
به منظور استخراج فرم فضای حالت، متغیرهای حالت استند سه درجه آزادی چهارپره به صورت زیر تعریف می‌شود:
\begin{equation}
	\begin{bmatrix}
		x_1\\x_2\\x_3\\x_4\\x_5\\x_6\\
	\end{bmatrix} = 
\begin{bmatrix}
	\phi\\ \theta \\ \psi \\ p\\ q\\ r
\end{bmatrix}
\end{equation}
معادلات ارائه شده به فرم زیر برای فضای حالت بازنویسی شدند:
\begin{equation}
	\dot x_1 = x_4 + x_5\sin(x_1)\tan(x_2) + x_6\cos(x_1)\tan(x_2)
\end{equation}
\begin{equation}
	\dot x_2 = x_5\cos(x_1)- x_6\sin(x_1)
\end{equation}
\begin{equation}
	\dot x_3 = (x_5\sin(x_1) + x_6\cos(x_1))\sec(x_2)
\end{equation}
\begin{equation}
	\dot x_4 = A_1\cos(x_2)\sin(x_1) + 
	A_2x_5x_6 + A_3\left(\omega_2^2-\omega_4^2\right)+
	A_4x_5\left(\omega_1-\omega_2+\omega_3-\omega_4\right) - \dfrac{x_4}{\lvert x_4\rvert}A_5
\end{equation}
\begin{equation}
	\dot x_5 = B_1\sin(x_2) + 
	B_2x_4x_6 + B_3\left(\omega_1^2-\omega_3^2\right)+
	B_4x_4\left(\omega_1-\omega_2+\omega_3-\omega_4\right) - \dfrac{x_5}{\lvert x_5\rvert}B_5
\end{equation}
\begin{equation}
	\dot x_6 = C_1x_4x_5 + 
	C_2\left(\omega_1^2-\omega_2^2+\omega_3^2-\omega_4^2\right) - \dfrac{x_6}{\lvert x_6\rvert}C_3
\end{equation}
ثابت‌های معادلات بالا  به صورت زیر تعریف می‌شوند:
\begin{align*}
	&A_1  = \dfrac{h_{cg}gm_{tot}}{m_{tot}h_{cg}^2+J_{11}}
	&A_2  = \dfrac{2m_{tot}h_{cg}^2+J_{22}-J_{33}}{m_{tot}h_{cg}^2+J_{11}}\quad
	&A_3  = \dfrac{bd_{cg}}{m_{tot}h_{cg}^2+J_{11}}
	&A_4  = \dfrac{J_R}{m_{tot}h_{cg}^2+J_{11}}\\
	& & A_5 = \dfrac{m_1g\mu r_x}{m_{tot}h_{cg}^2 + J_{11}} & &\\
	&B_1  = \dfrac{h_{cg}gm_{tot}}{m_{tot}h_{cg}^2+J_{22}}
	&B_2  = \dfrac{-2m_{tot}h_{cg}^2-J_{11}+J_{33}}{m_{tot}h_{cg}^2+J_{22}}\quad
	&B_3  = \dfrac{bd_{cg}}{m_{tot}h_{cg}^2+J_{22}}
	&B_4  = \dfrac{-J_R}{m_{tot}h_{cg}^2+J_{22}}\\
	& & B_5 = \dfrac{m_2g\mu r_y}{m_{tot}h_{cg}^2 + J_{22}} & &
\end{align*}
\begin{align*}
	C_1 =\dfrac{J_{11}-J_{22}}{J_{33}}\quad
	C_2 =\dfrac{d}{J_{33}}\quad
	C_3 = \dfrac{m_3g\mu r_z}{ J_{33}}
\end{align*}
به منظور شبیه‌سازی ، پارامترهای استند به صورت جدول 
\ref{parameterstable}
درنظر گرفته شده است که 
مقدار پارامترهای استند آزمایشگاه است.
\begin{table}[H]
	\caption {پارامترهای شبیه‌سازی استند چهارپره \cite{norian}} 
	\label{parameterstable}
	\begin{center}
		\begin{tabular}{ c c c }
			\hline
	پارامتر & واحد & مقدار پارامتر استند چهارپره    \\
			\hline

			$J_{11}$ & $kg.m^2$& $0.02839$    \\
			%\hline
			$J_{22}$ & $kg.m^2$& $0.03066$   \\
%			\hline
			$J_{33}$ & $kg.m^2$&$0.0439$    \\
			$J_{R}$   & $kg.m^2$&$4.4398\times 10^{-5}$\\
%			\hline
			$m_{tot}$ & $kg$& $1.074$    \\
			$m_{1}$ & $kg$& $1.272$    \\
			$m_{2}$ & $kg$& $1.074$    \\
			$m_{3}$ & $kg$& $1.693$    \\
			$d_{cg}$ & $m$ &$0.2$     \\
			$h_{cg}$ & $m$  &$0.02$   \\
			$r_{x}$ & $m$& $0.01$    \\
			$r_{y}$ & $m$& $0.01$    \\
			$r_{z}$ & $m$& $0.025$    \\
			$b$ &  $1$ &$3.13\times10^{-5}$   \\
			$d$ & $1$&$3.2\times10^{-6}$   \\
			$\mu_{s}$ & $1$& $0.003$    \\
			$\mu_{k}$ & $1$& $0.002$    \\
			$g$&$m/s^2$&$9.81$\\
			\hline
		\end{tabular}
	\end{center}
\end{table}


