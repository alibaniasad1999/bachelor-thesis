\section{معادله گشتاور}
به منظور استخراج معادلات حاکم بر حرکت دورانی چهارپره، از قوانین نیوتن اویلر استفاده می‌شود. 
معادله دیفرانسیلی اویلر برای یک پرنده حول مرکز ثقل آن در دستگاه مختصات بدنی به صورت زیر بیان می‌شود \cite{zipfel2000modeling}:
\begin{equation}\label{torque}
	\left[\dot{\boldsymbol{\omega}}^{BI}\right]^B = \left(\left[\boldsymbol J\right]^B\right)^{-1}
	\left(-\left[\boldsymbol \Omega^{BI}\right]^B\times\left(
	\left[\boldsymbol J\right]^B\left[\boldsymbol \omega^{BI}\right]^B+
	\left[\boldsymbol I_R\right]^B
	\right)+ \left[\boldsymbol m_b\right]^B\right)
\end{equation}
در رابطه
(\ref{torque})، عبارت 
$\left[\dot{\boldsymbol\omega}^{BI}\right]^B$
بیانگر بردار مشتق نرخ‌های زاویه‌ای چهارپره در دستگاه مختصات بدنی است. همچنین ماتریس 
$\left[\boldsymbol J\right]^B$
نشان‌دهنده گشتاورهای اینرسی چهارپره حول مرکز ثقل آن در دستگاه مختصات بدنی است که به دلیل تقارن چهارپره به صورت زیر درنظر گرفته
 می‌شود:
 \begin{equation}\label{Jmatrix}
 	\left[\boldsymbol J\right]^B = \begin{bmatrix}
 		J_{11} & 0 &0\\
 		0 & J_{22} & 0\\
 		0 & 0 & J_{33}
 	\end{bmatrix}
 \end{equation}
در رابطه 
(\ref{Jmatrix})، پارامترهای 
$J_{11}$،
$J_{22}$
و 
$J_{33}$
به ترتیب بیانگر گشتاور‌های اینرسی چهارپره حول محورهای 
$X^B$،
$Y^B$
و 
$Z^B$
دستگاه مختصات بدنی هستند. همچنین بردار 
$\left[\boldsymbol I_R\right]^B$
در رابطه‌ی 
(\ref{torque})
بیانگر مجموع تكانه زاویه‌ای کلی پره‌ها در دستگاه مختصات بدنی است. ازآنجا که، تكانه زاویه‌ای پره‌ها در راستای محور
$Z^B$
دستگاه مختصات بدنی است؛ در نتیجه 
$\left[\boldsymbol I_R\right]^B$
به صورت زیر حاصل می‌شود:
\begin{equation}\label{IR}
	\left[\boldsymbol I_R\right]^B = 
	\begin{bmatrix}
		0\\0\\l_R
	\end{bmatrix}
\end{equation}
در رابطه‌ی 
(\ref{IR})، 
$l_R$
بیانگر تكانه زاویه‌ای کلی پره‌ها در راستای محور
$Z^B$
دستگاه مختصات بدنی است که به صورت زیر حاصل می‌شود:
\begin{equation}\label{IRomega}
	l_R = J_R\omega_d
\end{equation}
در رابطه‌ی
(\ref{IRomega})، پارامتر
$J_R$
بیانگر ممان اینرسی هر یک از پره‌ها است. همچنین
$\omega_d$
نشان دهنده تفاضل نسبی سرعت‌های زاویه‌ای پره‌ها است که با توجه به شكل
(\ref{QuadAssum})
به صورت زیر تعریف می‌شود:
\begin{equation}\label{omega_d}
	\omega_d = -\omega_1 + \omega_2-\omega_3 + \omega_4
\end{equation}
همچنین 
$\left[\boldsymbol m_b\right]^B$
در رابطه‌ی
(\ref{torque})
برآیند گشتاورهای خارجی اعمالی به چهارپره، شامل 
گشتاورهای ناشی از آیرودینامیک پره‌ها و گشتاورهای ناشی از نیروی تكیه‌گاه است که در ادامه به آن پرداخته می‌شود.



