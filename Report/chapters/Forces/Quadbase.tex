\chapter{مدل‌سازی چهارپره}
 در این فصل به مدلسازی استند چهارپره آزمایشگاهی  پرداخته شده است. به این منظور، ابتدا فرضیات مربوط به 
 مدلسازی چهارپره بیان می‌شود. سپس معادلات حاکم بر حرکات 
 دورانی چهارپره بیان می‌شود. در ادامه به استخراج گشتاورهای خارجی اعمالی 
 به استند شامل گشتاورهای آیرودینامیکی ناشی از پره، گشتاور نیروی تکیه گاه و گشتاورهای ناشی از 
 اصطکاک بیرینگ‌ها پرداخته می‌شود. در گام بعد، معادله نهایی دینامیک دورانی استند 
 استخراج می‌شود. سپس، فرم فضای حالت استند آزمایشگاهی استخراج می‌شود. لازم به 
 توضیح است که فرم نهایی فضای حالت استند بدون درنظرگرفتن اصطکاک بیریناگ ها از منبع
 \cite{Abeshtan}
 آورده ‌شده ‌است که در آن منبع، مدل استخراج شده با اعمال ورودی های و شرایط اولیه مختلاف 
 اعتبارسنجی شده ‌است.