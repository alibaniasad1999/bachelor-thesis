\subsection{شبیه‌سازی کانال رول استند در حضور کنترل‌کننده LQDG}
در بخش
\ref{quadchanell_roll}
شبیه‌سازی کانال رول استند چهارپره انجام شد. در این بخش به بررسی عملکرد چهارپره در حضور کنترل‌کننده LQDG پرداخته می‌شود. کنترل‌کننده LQDG در بخش‌های
\ref{openloop_game}
و
\ref{closedloop_game}
بررسی شده است.
 در شبیه‌سازی برای بهینه‌سازی ضرایب وزنی از روش
TCACS \cite{Karimi2010}
استفاده شده است.
\begin{figure}[H]\label{lqdg_roll_fig}
	\includegraphics[width=12cm]{../Figures/MIL/LQDG/Roll/lqdg_roll.png}
	\centering
	\caption{عملكرد LQDG در کنترل زاويه رول (تعقیب ورودی صفر)}
\end{figure}
بر اساس خروجی شبیه‌سازی (شکل
\ref{lqdg_roll_fig})
،کانال رول در حضور کنترل‌کننده LQDG در کمتر از پنج ثانیه به تعادل می‌رسد اما دارای خطای ماندگار است ولی خطای مانگار آن نسبت به کنترل‌کننده بخش
\ref{roll_lqr_section}
کمتر است. به دلیل خطای ماندگار، در بخش
%LQIDG
انتگرالگیر به کنترل‌کننده اضافه می‌شود تا خطای مانگار استند را کم کند.
