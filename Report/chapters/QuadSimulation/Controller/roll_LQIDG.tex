%\section{شبیه‌سازی کانال رول استند در حضور کنترل‌کننده \lr{LQIDG}}\label{roll_lqidg_section_simulation}
%در بخش
%\ref{quadchanell_roll}
%شبیه‌سازی کانال رول استند چهارپره انجام شد.
 در این بخش به بررسی عملکرد چهارپره در حضور کنترل‌کننده \lr{LQIDG} پرداخته می‌شود. کنترل‌کننده \lr{LQIDG} در بخش
\ref{LQIDG}
بررسی شده‌است.
 در شبیه‌سازی برای بهینه‌سازی ضرایب وزنی \lr{LQIDG} از روش بهینه‌سازی
\lr{TCACS} \cite{Karimi2010}
استفاده شده‌است.
\begin{figure}[H]
	\includegraphics[width=.48\linewidth]{../Figures/MIL/LQIDG/Roll/lqidg_rollnn_.png}
	\centering
	\caption{عملكرد کنترل‌کننده \lr{LQIDG} در کنترل زاويه رول (تعقیب ورودی صفر)}
	\label{lqidg_roll_fig_simulation}
\end{figure}
\begin{figure}[H]
	\centering
	\subfigure[موتور شماره دو]{
		\centering
		\includegraphics[width=.45\linewidth]{../Figures/MIL/LQIDG/Roll/lqidg_roll_Omega_2nn_.png}
	}
	\subfigure[موتور شماره چهار]{
		\centering
		\includegraphics[width=.45\linewidth]{../Figures/MIL/LQIDG/Roll/lqidg_roll_Omega_4nn_.png}
	}
	\caption{‫‪فرمان کنترلی موتورها در کنترل زاویه رول (تعقیب ورودی صفر)}
\end{figure}

بر اساس خروجی شبیه‌سازی (شکل
\ref{lqidg_roll_fig_simulation})،
کانال رول در حضور کنترل‌کننده \lr{LQIDG} در حدود پنج ثانیه به تعادل می‌رسد و خطای ماندگار آن به دلیل وجود انتگرال‌گیر در حدود صفر است.