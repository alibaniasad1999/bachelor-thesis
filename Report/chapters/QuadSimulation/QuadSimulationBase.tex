\chapter{شبیه‌سازی  در محیط سیمولینک}
سیمولینک\LTRfootnote{Simulink} یک ابزار شبیه‌سازی همراه با نرم‌افزار متلب است.
با استفاده از سیمولینک می‌توان رفتار یک سیستم را بدون نیاز به ساخت آن را تحلیل نمود. در نتیجه با استفاده از سیمولینک می‌توان علاوه بر صرفه‌جویی در هزینه و زمان به بررسی تأثیر اغتشاشات یا سایر عوامل ورودی بر عملکرد یک سیستم پرداخت. همچنین شبیه‌سازی سیستم‌ها این توانایی را در اختیار می‌گذارد تا عکس‌العمل یک سیستم در صورت تغییر پارامترهای ورودی آن به خوبی شناخته شود. سیمولینک به صورت یک کتابخانه در نرم‌افزار متلب\LTRfootnote{MATLAB} عرضه شده‌است که شبیه‌سازی توسط بلوک‌های این کتابخانه به صورت دیاگرام‌های بلوکی انجام می‌شود.