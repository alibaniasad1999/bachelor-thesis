\section{اصلاح پارامتر‌های  استند چهارپره در شبیه‌سازی حلقه باز}
در بخش
(\ref{spacestate})
فرم فضای حالت استند چهارپره استخراج شد. در بخش
(\ref{quadall3})
شبیه‌سازی استند چهارپره انجام شد و در بخش
(\ref{quadchanell})
کانال‌های مختلف استند سه درجه آزادی چهارپره شبیه‌سازی شد.
در این بخش با استفاده از شبیه‌سازی‌های چهارپره در محیط سیمولینک و داده‌های خروجی  از استند چهارپره، پارامترهای استند چهارپره اصلاح می‌شوند.

برای اصلاح پارامترهای استند چهارپره از جعبه‌ابزار
\lr{Parameter Estimator}
موجود در سیمولینک
استفاده شده است.
این جعبه ابزار با استفاده از داده‌های خروجی در واقعیت و داده‌های خروجی در سیمولینک اقدام به اصلاح پارامتر می‌کند، به‌صورتی که خروجی شبیه‌سازی و خروجی داده از پلنت واقعی تا حد ممکن به هم نزدیک شودند.

%\newline


\begin{figure}[H]
	\includegraphics[width=12cm]{../../Figures/QuadSimulation/ParameterEstimation/PS_icon.png}
	\centering
	\caption{نماد جعبه‌ابزار
	\lr{Parameter Estimator}
در سیمولینک}
\end{figure}
در شکل
(\ref{PS})
نمایی از جعبه‌ابزار
\lr{Parameter Estimator}
 اورده شد است.





\begin{figure}[H]
	\includegraphics[width=12cm]{../../Figures/QuadSimulation/ParameterEstimation/PS_app.png}
	\centering
	\caption{جعبه‌ابزار
	\lr{Parameter Estimator}}
	\label{PS}
\end{figure}

