\section{شبیه‌سازی استند سه درجه آزادی در محیط سیمولینک}\label{quadall3}
در این بخش به بررسی و شبیه‌سازی مدل دینامیکی استند سه درجه آزادی پرداخته شده است. در بخش \ref{spacestate} فرم فضای حالت استند چهارپره استخراج شد. در شبیه‌سازی نیز از همین روابط استخراج‌شده، استفاده شده است. مدل شبیه‌سازی‌شده از استند (شکل \ref{quadsimulink}) دارای چهار ورودی سرعت دورانی موتورها  و دارای سه خروجی زوایای رول ($\phi$)، پیچ ($\theta$)، یاو ($\psi$) و  سه سرعت زاویه‌ای
 \lr{p},
\lr{q}
 و 
\lr{r} 
 است.
 
 	\hspace*{0.5cm}
\begin{figure}[H]
	\includegraphics[width=16cm]{../Figures/QuadSimulation/Stand_Model.png}
	\centering
%	\vspace*{-15mm}
	\caption{مدل استند چهارپره شبیه‌سازی شده در سیمولینک و نمایش ورودی و خروجی‌های مدل}
	\label{quadsimulink}
\end{figure}
نمایی از داخل بلوک
\lr{Quacopter 3DOF Nonlinear System}
در شکل \ref{Quad3DOF} آورده شده‌است. این بلوک بر اساس معادلات فضای حالت بدست آمده در بخش
\ref{spacestate}
ساخته شده‌است.
%\newline
%\newline
	\hspace*{1.5cm}
\begin{figure}[H]
	\includegraphics[width=16cm]{../Figures/QuadSimulation/Integrator.png}
	\centering
%	\vspace*{-15mm}
	\caption{مدل استند چهارپره شبیه‌سازی شده در سیمولینک و نمایش ورودی و خروجی‌های مدل}
	\label{Quad3DOF}
\end{figure}
خروجی بلوک
\lr{Quad System}،
مشتق متغیرهای حالت 
$\dot X$
است. بردار حالت سیستم با استفاده از بلوک انتگرال‌گیر (بلوک زرد رنگ در شکل \ref{Quad3DOF})
%از خروجی بلوک بر اساس شرایط اولیه استند انتگرال گرفته می‌شود و وضعیت استند
 (زاویه‌های رول ($\phi$)، پیچ ($\theta$)، یاو ($\psi$) و سرعت‌های زاویه‌ای‌
 \lr{p},
\lr{q}
و 
\lr{r})
به دست می‌آید.

در داخل بلوک
\lr{Quad System}،
شش بلوک دیگر قرار دارد که تعدادی از آن‌ها دارای ورودی $X$ و تعدادی دیگر دارای ورودی $X$ و $\omega$ هستند. مجموع خروجی این شش بلوک $\dot X$ است که در توضیحات بلوک
\lr{Quad System}،
 نیز به آن اشاره شد.
نمایی از داخل بلوک
\lr{Quad System}
در شکل \ref{all-six} آورده شده‌است.
\begin{figure}[H]
	\includegraphics[width=16cm]{../Figures/QuadSimulation/all-six.png}
	\centering
%	\vspace*{-15mm}
	\caption{نمایی از داخل بلوک \lr{Quad System}}
	\label{all-six}
\end{figure}