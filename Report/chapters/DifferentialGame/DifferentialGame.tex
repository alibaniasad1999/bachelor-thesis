\chapter{کنترل‌کننده مبتنی بر بازی دیفرانسیلی}
در تئوری بازی‌ها، بازی‌های دیفرانسیل مجموعه‌ای از مسائل مربوط به مدل‌سازی و تحلیل در چارچوب یک سامانه دینامیکی هستند. به طور خاص، یک متغیر یا متغیرهای حالت در طول زمان مطابق با یک معادله دیفرانسیل تکامل رفتار می‌کنند
\cite{diff_game}.
% تمامی توضیحات و روابط  از منبع 
% \cite{article1}
%آورده شده است. در این فصل حالت حلقه‌باز\LTRfootnote{Opne Loop}
% و حالت همراه با بازخورد\LTRfootnote{Feedback} بررسی می‌شود.
 % یا شده است؟
 این پروژه حالت دو بازیکن را بررسی می‌کند. در این مسئله برای یک سامانه خطی پیوسته با معالات حالت:
 \begin{equation}\label{system_dynamic}
 	\begin{split}
 		 	&\boldsymbol{\dot x}(t) = \boldsymbol{Ax}(t) + \boldsymbol{B_1u_1}(t) + \boldsymbol{B_2u_2}(t)%, \quad \boldsymbol{x}(0) = \boldsymbol{x}_0%
 		\\
 		&\boldsymbol{y}(t) = \boldsymbol{Cx}(t) + \boldsymbol{D_1u_1}(t) + \boldsymbol{D_2u_2}(t)
 	\end{split}
 \end{equation}
که در رابطه (\ref{system_dynamic})
$\boldsymbol x$, $\boldsymbol y$, $\boldsymbol{u_1}$
و
$\boldsymbol{u_2}$
به ترتیب بیانگر بردار حالت، بردار خروجی، بردار ورودی بازیکن اول و بردار ورودی بازیکن دوم هستند. همچنین، 
$\boldsymbol A$, $\boldsymbol{B_1}$, $\boldsymbol {B_2}$, $\boldsymbol C$, $\boldsymbol {D_1}$
و
$\boldsymbol{D_2}$
به ترتیب بیانگر ماتریس حالت، ماتریس ورودی بازیکن اول، ماتریس ورودی بازیکن دوم، ماتریس خروجی، ماتریس فیدفوروارد بازیکن اول و ماتریس فیدفوروارد بازیکن دوم هستند
\cite{mct}.
بر اساس رابطه (\ref{system_dynamic}) دینامیک سیستم تحت تاثیر هر دو بازیکن قرار می‌گیرد. در اینجا ممکن است تلاش  بازیکن اول موجب دور شدن بازیکن دوم از هدف شود و یا برعکس.  این پروژه حالت همکاری دو بازیکن را بررسی نمی‌کند و دو بازیکن در تلاش برای کم کردن تابع هزینه خود و زیاد کردن تابع هزینه بازیکن مقابل هستند.

 
  فرض شده  که تابع هزینه برای هر بازیکن در زمان $t \in [0, T]$ به صورت مربعی\LTRfootnote{Quadratic Cost Function} است.
  هدف اصلی کم کردن تابع هزینه برای بازیکنان است. تابع هزینه برای بازیکن شماره \lr{i}  (این مسئله شامل دو بازیکن است) به فرم رابطه (\ref{cost}) نوشته می‌شود.

 \begin{equation}\label{cost}
 	J_i(u_1, u_2) = \int_{0}^{T}\left( \boldsymbol{x} ^\mathrm{T}(t) \boldsymbol{Q_i} \boldsymbol{x}(t)+
 	 \boldsymbol{u_i} ^\mathrm{T}(t) \boldsymbol{R_{ii}} \boldsymbol{u_i}(t)+
 	 \boldsymbol{u_j} ^\mathrm{T}(t)\boldsymbol{ R_{ij} u_j}(t)
 	\right)dt+
 	\boldsymbol{ x} ^\mathrm{T}(T)\boldsymbol{ H_i}\boldsymbol{ x}(T) 
  \end{equation}
در رابطه
(\ref{cost})
$\boldsymbol{Q_i}$, $\boldsymbol{R_{ii}}$, $\boldsymbol{R_{ij}}$
و
$\boldsymbol{H_i}$
به ترتیب بیانگر اهمیت میزان اهمیت انحراف متغیرهای حالت مقادیر مطلوب برای بازیکن شماره \lr{i} ،میزان تلاش کنترلی بازیگر شماره \lr{i} ، میزان تلاش کنترلی بازیگر شماره \lr{j}  و میزان اهمیت انحراف متغیرهای حالت مقادیر مطلوب در حالت پایانی برای بازیکن شماره \lr{i}  هستند.
در اینجا ماتریس‌های 
$\boldsymbol{Q_i}$ ، $\boldsymbol{R_{ii}}$
و
$\boldsymbol{H}$
متقارن فرض شده‌اند و ماتریس 
$\boldsymbol{R_{ii}}$
به صورت مثبت معین ($\boldsymbol{R_{ii}}>0$)
فرض شده است.
%دینامیک سیستم تحت تاثیر هر دو بازیکن قرار می‌گیرد. در اینجا دینامیک سیستم به فرم رابطه (\ref{system_dynamic}) در نظر گرفته شده ‌است.

%در رابطه 
%(\ref{system_dynamic})،
%$\boldsymbol{u_1}$
%برابر با تلاش کنترلی  بازیکن اول است. در اینجا ممکن است تلاش  بازیکن اول موجب دور شدن بازیکن دوم از هدف شود و یا برعکس.  این پروژه حالت همکاری دو بازیکن را بررسی نمی‌کند و دو بازیکن در تلاش برای کم کردن تابع هزینه خود و زیاد کردن تابع هزینه بازیکن مقابل هستند.
در بخش
\ref{LQDG}
به معرفی
کنترل‌کننده مربعی خطی مبتنی بر بازی دیفرانسیلی و در بخش
\ref{LQIDG}
به معرفی
کنترل‌کننده مربعی خطی انتگرالی مبتنی بر بازی دیفرانسیلی پرداخته می‌شود.
