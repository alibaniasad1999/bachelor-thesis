\section{بازی حلقه‌باز}
در این حالت فرض شده است که تمامی بازیکنان در زمان 
$t \in [0, T]$
فقط اطلاعات شرایط اولیه و مدل سیستم را دارند. این فرض به این صورت تفسیر می‌شود که دو بازیکن همزمان حرکت خود را در انتخاب می‌کنند. در این حالت امکان هماهنگی بین دو بازیکن وجود ندارد. تعادل نش در ادامه تعریف شده ‌است.
\شروع{قضیه} 
به مجموعه‌ای از حرکات قابل قبول 
$(\boldsymbol{u_1}^*,  \boldsymbol{u_2}^*)$
یک \مهم{تعادل نش} برای بازی می‌گویند اگر تمامی حرکات قابل قبول 
$(\boldsymbol{u_1},  \boldsymbol{u_2})$
از نامساوی (\ref{nash_lqe}) پیروی کنند.
\begin{equation}\label{nash_lqe}
	J_1(\boldsymbol{u_1}^*, \boldsymbol{u_2}^*)\leq J_1(\boldsymbol{u_1}, \boldsymbol{u_2}^*) \text{\rm{ and }}
	J_2(\boldsymbol{u_1}^*, \boldsymbol{u_2}^*)\leq 
	J_2(\boldsymbol{u_1}^*, \boldsymbol{u_2})
\end{equation}
\پایان{قضیه}
در اینجا قابل قبول بودن به‌معنی آن است که
$\boldsymbol{u_i}(.)$
به یک مجموعه محدود حرکات تعلق دارد، این مجموعه‌ی حرکات که بستگی به اطلاعات بازیکنان از بازی دارد، مجموعه‌ای از راهبردهایی است که بازیکنان ترجیح می‌دهند برای کنترل سیستم انجام دهند و سیستم 
(\ref{system_dynamic})
باید یک جواب منحصر به فرد داشته باشد. 


تعادل نش به گونه‌ای تعریف می‌شود که هیچ یک از بازیکنان انگیزه‌ی یک طرفه برای انحراف از بازی ندارند. قابل ذکر است که نمی‌توان انتظار داشت که یک تعادل نش منحصر به فرد وجود داشته باشد. به هر حال به راحتی می‌توان تایید کرد که حرکات
$(\boldsymbol{u_1}^*, \boldsymbol{u_2}^*)$
یک تعادل نش برای بازی با تابع هزینه
$J_i,~ (i = 1, 2)$
است.
 %اگر تعادل نش برای تابع هزینه قسمت قبل برقرار باشد برای تابع هزینه
%$\alpha_iJ_i~ i = 1, 2, ~\alpha_i>0$
%نیز برقرار است.


برای سادگی از نمادسازی 
$\boldsymbol{S_i} := \boldsymbol{B_iR_{ii}}^{-1}\boldsymbol{B_i}^\mathrm{T}$
استفاده شده‌است. در اینجا فرض شده است که زمان $T$ محدود است.
\شروع{قضیه} \label{openlooptheorm}
ماتریس
$\boldsymbol{M}$ را در نظر بگیرید:
\begin{equation}
	\boldsymbol{M} :=
	\begin{bmatrix}
		\boldsymbol{A} & -\boldsymbol{S_1} & -\boldsymbol{S_2}\\
		-\boldsymbol{Q_1} & -\boldsymbol{A}^\mathrm{T}& \boldsymbol{0}\\
		-\boldsymbol{Q_2} & \boldsymbol{0} & -\boldsymbol{A}^\mathrm{T}
	\end{bmatrix}
\end{equation}
فرض شده ‌است که دو معادله دیفرانسیلی ریکاتی
(\ref{riccati_teorm})، 
 در بازه
$[0, T]$
جواب متقارن دارند.
\begin{equation}\label{riccati_teorm}
	\boldsymbol{\dot{K}_i}(t) = -\boldsymbol{A}^\mathrm{T}\boldsymbol{K_i}(t)-\boldsymbol{K_i}(t)\boldsymbol{A}+\boldsymbol{K_i}(t)\boldsymbol{S_iK_i}(t)-\boldsymbol{Q_i},\quad \boldsymbol{K_i}(T) = \boldsymbol{H_i},\quad i = 1, 2
\end{equation}
\newpage
بازی دیفرانسیل خطی درجه دوم دو نفره\LTRfootnote{the two player linear quadratic differential game} تعادل نش حلقه‌باز در هر شرایط اولیه $\boldsymbol{X_0}
$
دارد اگر ماتریس
\begin{equation}
	\boldsymbol{H}(T) := \begin{bmatrix}
		\boldsymbol{I}&0&0
	\end{bmatrix}
e^{-\boldsymbol{M}T}
\begin{bmatrix}
	\boldsymbol{I}
	\\ \boldsymbol{H_{1}}
	\\ \boldsymbol{H_{2}}
\end{bmatrix}
\end{equation}
معکوس‌پذیر‌ باشد
 \cite{article1}
 .
\پایان{قضیه}
در معادلات بالا تلاش کنترلی برای هر بازیکن به فرم رابطه \ref{openloop_u} تعریف شده است.
\begin{equation}\label{openloop_u}
	\boldsymbol{u_i}(t) = -\boldsymbol{R_{ii}}\boldsymbol{B_i}^\mathrm{T}\boldsymbol{x}(t),\quad i = 1, 2
\end{equation}
در آخر با استفاده از قضیه
 \ref{openlooptheorm}
با حل دو معادله کوپل ریکاتی دیفرانسیلی می‌توان به جواب رسید.
\begin{align}
	\boldsymbol{\dot{K}_1} &= -\boldsymbol{A}^\mathrm{T}\boldsymbol{K_1} - \boldsymbol{K_1A} - \boldsymbol{Q_1} +\boldsymbol{K_1S_1K_1} + \boldsymbol{K_1S_2K_2};\quad \boldsymbol{K_1}(T) = \boldsymbol{H_1}\\
	\boldsymbol{\dot{K}_2} &= -\boldsymbol{A}^\mathrm{T}\boldsymbol{K_2} - \boldsymbol{K_2A} - \boldsymbol{Q_2} +\boldsymbol{K_2S_2K_2} + \boldsymbol{K_2S_1K_1};\quad \boldsymbol{K_2}(T) = \boldsymbol{H_2}
\end{align}

