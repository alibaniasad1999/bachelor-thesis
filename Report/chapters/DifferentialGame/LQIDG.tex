\section{بازی همراه با بازخورد}\label{closedloop_game}
تفاوت بازی همراه با بازخورد\LTRfootnote{The Feeback Game} با بازی حلقه‌باز در این است که بازیکنان در هر لحظه از بازی بازخورد می‌گیرند و متناسب با بازخورد رفتار می‌کنند. این بازخورد ممکن است باعث شود یک بازیکن انگیزه پیدا کند که از بازی انحراف پیدا کند در حالی که این اتفاق در بازی حلقه‌باز رخ نمی‌دهد. این اتفاق منجر به یک راه حل تعادلی دیگر می‌شود. 
%از طرف دیگر راه حل تعادلی نباید در طول بازی خودش را با بازیکنان سازگار کند.
با توجه به اینکه سیستم خطی است، می‌توان استدلال کرد که جواب بهینه به صورت تابعی خطی از وضعیت سیستم است \cite{article1}
.
% این بدین مفهوم است که تعادل نش باید در فضای ذکر شده باشد. فضای راهبردی
% استراتژی
% به فرم رابطه
% \ref{NashSpace}
%
%\begin{equation}\label{NashSpace}
%	\boldsymbol{\Gamma^{lfb}_i} :‌= \left\{\boldsymbol{u_i}(0, T)\vert \boldsymbol{u_i}(t) = \boldsymbol{F_i}(t)\boldsymbol{x}(t) ,~ i = 1, 2\right\}
%\end{equation}
%تعریف می‌شود. در رابطه
% \ref{NashSpace}
%$\boldsymbol{F_i}(.)$
%قسمتی از یک تابع است. حرکات تعادل نش
%$(\boldsymbol{u_1}^*, \boldsymbol{u_2}^*)$
%در فضای استراتژی 
%$\boldsymbol{\Gamma^{lfb}_1}\times\boldsymbol{\Gamma^{lfb}_2}$
%است.
\شروع{قضیه} 
مجموعه‌ی حرکات کنترلی 
$\boldsymbol{u_i}^*(t)=\boldsymbol{F_i}^*(t)\boldsymbol{x}(t)$
تشکیل شده‌است از بازخورد خطی تعادل نش اگر
\begin{equation*}
	J_1(\boldsymbol{u_1}^*, \boldsymbol{u_2}^*)\leq J_1(\boldsymbol{u_1}, \boldsymbol{u_2}^*)\text{\rm{ and }}
	J_2(\boldsymbol{u_1}^*,\boldsymbol{ u_2}^*)\leq J_2(\boldsymbol{u_1}^*, \boldsymbol{u_2})
\end{equation*}
برای هر 
$\boldsymbol{u_i}\in \boldsymbol{\Gamma^{lfb}_i}$
برقرار باشد.
\پایان{قضیه}
\شروع{قضیه}
بازی دیفرانسیلی خطی درجه دوم دو نفره برای هر شرایط اولیه، تعادل نش خطی بازخورد دارد اگر و فقط اگر مجموعه معادلات کوپل ریکاتی

\begin{align}
	\begin{split}
		\boldsymbol{\dot{K}_1}(t) &= -(\boldsymbol{A}-\boldsymbol{S_2}\boldsymbol{K_2}(t))^\mathrm{T}\boldsymbol{K_1}(t)-\boldsymbol{K_1}(t)(\boldsymbol{A}-\boldsymbol{S_2K_2}(t))+
		\boldsymbol{K_1}(t)\boldsymbol{S_1K_1}(t)-\boldsymbol{Q_1}\\
		\quad \boldsymbol{K_1}(T) &= \boldsymbol{H_1}
	\end{split}\\
	\begin{split}
		\boldsymbol{\dot{K}_2}(t) &= -(\boldsymbol{A}-\boldsymbol{S_1}\boldsymbol{K_1}(t))^\mathrm{T}\boldsymbol{K_2}(t)-\boldsymbol{K_2}(t)(\boldsymbol{A}-\boldsymbol{S_1K_1}(t))+
\boldsymbol{K_2}(t)\boldsymbol{S_2K_2}(t)-\boldsymbol{Q_2}\\
\quad \boldsymbol{K_2}(T) &= \boldsymbol{H_2}
	\end{split}
\end{align}
در بازه زمانی 
$[0, T]$
جواب متقارن داشته ‌باشند (برای سادگی 
$\boldsymbol{S_{12}}=\boldsymbol{S_{21}} =\boldsymbol 0 $
فرض شده است).
در این حالت دارای تعادل منحصر به فرد است. حرکت‌های تعادل به فرم رابطه
\ref{nash_action}
است.
\begin{equation}\label{nash_action}
	\boldsymbol{u_i}^*(t) = -\boldsymbol{R_{ii}B_i}^T\boldsymbol{K_i}(T)\boldsymbol{x}(T),~i = 1, 2
\end{equation}
\پایان{قضیه}
