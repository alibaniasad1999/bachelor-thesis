\chapter{شبیه‌سازی استند سه درجه آزادی چهارپره در محیط سیمولینک}
سیمولینک\LTRfootnote{Simulink} یک ابزار شبیه‌سازی همراه با نرم‌افزار متلب\LTRfootnote{MATLAB} است.
با استفاده از سیمولینک می‌توان یک سامانه دینامیکی را شبیه‌سازی کرد. بنابراین، به کمک این نرم‌افزار می‌توان رفتار سامانه‌های دینامیکی را بدون ساخت آن‌ها تحلیل کرد. علاوه بر این، به کمک شبیه‌سازی می‌توان رفتار سامانه را در شرایط مختلف مطالعه کرد؛ شرایطی که فراهم کردن آن در دنیای واقعی ممکن است هزینه‌بر و  یا دشوار باشد. سیمولینک به‌صورت یک افزونه در نرم‌افزار متلب عرضه شده‌است که شبیه‌سازی در محیط آن به‌صورت دیاگرام‌های بلوکی انجام می‌شود. 
در بخش
\ref{model_base}
 مراحل طراحی مدل‌مبنا و سپس، در بخش‌های
\ref{quadall3}
و
\ref{parameeter_estimation_section}
به بررسی شبیه‌سازی و اصلاح پارامتر استند سه درجه آزادی چهارپره پرداخته می‌شود.