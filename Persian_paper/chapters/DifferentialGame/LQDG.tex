
 %اگر تعادل نش برای تابع هزینه قسمت قبل برقرار باشد برای تابع هزینه
%$\alpha_iJ_i~ i = 1, 2, ~\alpha_i>0$
%نیز برقرار است.

\subsection{کنترل‌کننده مربعی خطی مبتنی بر بازی دیفرانسیلی}\label{LQDG}
%برای سادگی از نمادسازی 
%$\boldsymbol{S_i} := \boldsymbol{B_iR_{ii}}^{-1}\boldsymbol{B_i}^\mathrm{T}$
%استفاده شده‌است. در اینجا فرض شده است که زمان $T$ محدود است.
%\شروع{قضیه} \label{openlooptheorm}
%ماتریس
%$\boldsymbol{M}$ را در نظر بگیرید:
%\begin{equation}
%	\boldsymbol{M} :=
%	\begin{bmatrix}
%		\boldsymbol{A} & -\boldsymbol{S_1} & -\boldsymbol{S_2}\\
%		-\boldsymbol{Q_1} & -\boldsymbol{A}^\mathrm{T}& \boldsymbol{0}\\
%		-\boldsymbol{Q_2} & \boldsymbol{0} & -\boldsymbol{A}^\mathrm{T}
%	\end{bmatrix}
%\end{equation}
%فرض شده ‌است که دو معادله دیفرانسیلی ریکاتی
%(\ref{riccati_teorm})، 
% در بازه
%$[0, T]$
%جواب متقارن دارند.
%\begin{equation}\label{riccati_teorm}
%	\boldsymbol{\dot{P}_i}(t) = -\boldsymbol{A}^\mathrm{T}\boldsymbol{P_i}(t)-\boldsymbol{P_i}(t)\boldsymbol{A}+\boldsymbol{P_i}(t)\boldsymbol{S_iP_i}(t)-\boldsymbol{Q_i},\quad \boldsymbol{P_i}(T) = \boldsymbol{H_i},\quad i = 1, 2
%\end{equation}
%\newpage
%بازی دیفرانسیل خطی درجه دوم دو نفره\LTRfootnote{the two player linear quadratic differential game} تعادل نش حلقه‌باز در هر شرایط اولیه $\boldsymbol{X_0}
%$
%دارد اگر ماتریس
%\begin{equation}
%	\boldsymbol{H}(T) := \begin{bmatrix}
%		\boldsymbol{I}&0&0
%	\end{bmatrix}
%e^{-\boldsymbol{M}T}
%\begin{bmatrix}
%	\boldsymbol{I}
%	\\ \boldsymbol{H_{1}}
%	\\ \boldsymbol{H_{2}}
%\end{bmatrix}
%\end{equation}
%معکوس‌پذیر‌ باشد
% \cite{article1}
% .
%\پایان{قضیه}
برای یک سامانه خطی پیوسته با معادلات حالت:
 \begin{equation}\label{systemlqdg}
 	\begin{split}
	&\boldsymbol{\dot{\mathrm{x}}}(t) = \boldsymbol{\mathrm{Ax}}(t) + \boldsymbol{\mathrm{B_1u_1}}(t) + \boldsymbol{\mathrm{B_2u_2}}(t)%, \quad \boldsymbol{x}(0) = \boldsymbol{x}_0%
	\\
	&\boldsymbol{\mathrm{y}}(t) = \boldsymbol{\mathrm{Cx}}(t) + \boldsymbol{\mathrm{D_1u_1}}(t) + \boldsymbol{\mathrm{D_2u_2}}(t)
\end{split}
\end{equation}
فرمان کنترلی بهینه \lr{LQDG} بازیکن شماره \lr{i} به‌صورت رابطه
(\ref{openloop_u})
محاسبه می‌شود.
\begin{equation}\label{openloop_u}
	\boldsymbol{\mathrm{u_i}}(t) = -\boldsymbol{\mathrm{R_{ii}}}^{-1}\boldsymbol{\mathrm{B_i}}^\mathrm{T}\boldsymbol{\mathrm{P_{i}}}(t)\boldsymbol{\mathrm{x}}(t) = -\boldsymbol{\mathrm{k_{i}}}(t)\boldsymbol{\mathrm{x}}(t),\quad i = 1, 2
\end{equation}
که در رابطه 
(\ref{openloop_u})
،ضریب  $\boldsymbol{K}(t)$ بیانگر بهره بازخورد بهینه است. این بهره به گونه‌ای محاسبه می‌شود که تابع هزینه مربعی بازیکن شماره \lr{i} با فرض بدترین حرکت سایر بازیکنان کمینه شود. تابع هزینه بازیکن شماره \lr{i} در زیر آورده شده‌است.
 \begin{equation}\label{cost}
	J_i( \boldsymbol{\mathrm{u_1}},  \boldsymbol{\mathrm{u_2}}) = \int_{0}^{T}\left( \boldsymbol{\mathrm{x}} ^\mathrm{T}(t) \boldsymbol{\mathrm{Q_i}} \boldsymbol{\mathrm{x}}(t)+
	\boldsymbol{\mathrm{u_i}} ^\mathrm{T}(t) \boldsymbol{\mathrm{R_{ii}}} \boldsymbol{\mathrm{u_i}}(t)+
	\boldsymbol{\mathrm{u_j}} ^\mathrm{T}(t)\boldsymbol{\mathrm{ R_{ij} u_j}}(t)
	\right)\mathrm{d}t
	% 	\boldsymbol{ x} ^\mathrm{T}(T)\boldsymbol{ H_i}\boldsymbol{ x}(T) 
\end{equation}
در رابطه
در رابطه 
(\ref{openloop_u}),
ماتریس $\boldsymbol{P_{i}}(t)$ بیانگر پاسخ معادله کوپل ریكاتی\LTRfootnote{Coupled Riccati Differential Equations}
 زیر است
 \cite{diff_game}:
\begin{equation}\label{coupled_riccatti_LQDG}
\begin{split}
	\boldsymbol{\dot{\mathrm{P}}_1}(t) &= -\boldsymbol{\mathrm{A}}^\mathrm{T}\boldsymbol{\mathrm{P_1}}(t) - \boldsymbol{\mathrm{P_1}}(t)\boldsymbol{\mathrm{A}} - \boldsymbol{\mathrm{Q_1}} +\boldsymbol{\mathrm{P_1}}(t)\boldsymbol{\mathrm{S_1}}(t)\boldsymbol{\mathrm{P_1}}(t) + \boldsymbol{\mathrm{P_1}}(t)\boldsymbol{\mathrm{S_2}}(t)\boldsymbol{\mathrm{P_2}}(t)\\
	\boldsymbol{\dot{\mathrm{P}}_2}(t) &= -\boldsymbol{\mathrm{A}}^\mathrm{T}\boldsymbol{\mathrm{P_2}}(t) - \boldsymbol{\mathrm{P_2}}(t)\boldsymbol{\mathrm{A}} - \boldsymbol{\mathrm{Q_2}} +\boldsymbol{\mathrm{P_2}}(t)\boldsymbol{\mathrm{S_2}}(t)\boldsymbol{\mathrm{P_2}}(t) + \boldsymbol{\mathrm{P_2}}(t)\boldsymbol{\mathrm{S_1}}(t)\boldsymbol{\mathrm{P_1}}(t)
\end{split}
\end{equation}
برای سادگی از نمادسازی
$\boldsymbol{S_i} := \boldsymbol{B_iR_{ii}}^{-1}\boldsymbol{B_i}^\mathrm{T}$
  استفاده شده‌است. 


