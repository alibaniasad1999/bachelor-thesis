\chapter{مدل‌سازی چهارپره}
 در این فصل به مدل‌سازی استند چهارپره آزمایشگاهی  پرداخته شده‌است. به این منظور، ابتدا فرضیات مربوط به 
 مدل‌سازی چهارپره در بخش
\ref{sec_modelassum}
 بیان می‌شود. سپس، در بخش
 \ref{sec:moment}
 معادلات حاکم بر حرکات دورانی چهارپره و در زیر بخش‌های
 \ref{sec:aeromoment},
 \ref{sec:edgemoment},
 \ref{sec:friction}
 و
 \ref{sec:mgmoent}
%  بیان می‌شود. در ادامه،
   به استخراج گشتاورهای خارجی اعمالی 
 به استند شامل گشتاورهای آیرودینامیکی ناشی از پره، گشتاور نیروی تکیه‌گاه، گشتاورهای ناشی از 
 اصطکاک بیرینگ‌ها و  گشتاورهای ناشی از جرم استند پرداخته می‌شود. در گام بعد، در بخش
 \ref{sec:finalrotate}
 معادله نهایی دینامیک دورانی استند 
 استخراج می‌شود. سپس، فرم فضای حالت استند آزمایشگاهی در بخش
 \ref{spacestate}
 استخراج می‌شود. لازم به 
 توضیح است که فرم نهایی فضای حالت استند بدون درنظرگرفتن اصطکاک بیرینگ‌ها از منبع
 \cite{Abeshtan}
 آورده ‌شده‌است که در آن منبع، مدل استخراج‌شده با اعمال ورودی‌ها و شرایط اولیه مختلف 
 اعتبارسنجی شده‌‌است.