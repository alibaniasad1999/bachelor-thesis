%\subsection{فرم خطی فضای حالت کانال‌های چهارپره}
%در این قسمت، با توجه به فضای حالت  به دست آمده در بخش
%\ref{spacestate}،
%چهارپره حول نقطه کار خطی‌سازی می‌شود.
%\begin{equation*}
%	\sigma_1 = \omega_2^2-\omega_4^2,\quad \sigma_2 = \omega_1^2-\omega_3^2,
%	\quad \sigma_3 = \omega_1^2-\omega_2^2+\omega_3^2-\omega_4^2,\quad \sigma_4 = \omega_1-\omega_2+\omega_3-\omega_4
%\end{equation*}
%\begin{equation*}
%	\boldsymbol a = \begin{bmatrix}
%		%		x_4 + x_5\sin(x_1)\tan(x_2) + x_6\cos(x_1)\tan(x_2)\\
%		%		x_5\cos(x_1)- x_6\sin(x_1)\\
%		%		(x_5\sin(x_1) + x_6\cos(x_1))\sec(x_2)\\
%		%		A_1\cos(x_2)\sin(x_1) + 
%		%		A_2x_5x_6 + A_3\left(\omega_2^2-\omega_4^2\right)+
%		%		A_4x_5\left(\omega_1-\omega_2+\omega_3-\omega_4\right)- \dfrac{x_4}{\lvert x_4\rvert}A_5+A_6\cos(x_1)\\
%		%		B_1\sin(x_2) + 
%		%		B_2x_4x_6 + B_3\left(\omega_1^2-\omega_3^2\right)+
%		%		B_4x_4\left(\omega_1-\omega_2+\omega_3-\omega_4\right)- \dfrac{x_5}{\lvert x_5\rvert}B_5 + B_6\cos(x_2)\\
%		%		C_1x_4x_5 + 
%		%		C_2\left(\omega_1^2-\omega_2^2+\omega_3^2-\omega_4^2\right)- \dfrac{x_6}{\lvert x_6\rvert}C_3
%		x_4 + x_5\sin(x_1)\tan(x_2) + x_6\cos(x_1)\tan(x_2)\\
%		x_5\cos(x_1)- x_6\sin(x_1)\\
%		(x_5\sin(x_1) + x_6\cos(x_1))\sec(x_2)\\
%		A_1\cos(x_2)\sin(x_1) + 
%		A_2x_5x_6 + A_3\sigma_1+
%		A_4x_5\sigma_4- \dfrac{x_4}{\lvert x_4\rvert}A_5+A_6\cos(x_1)\\
%		B_1\sin(x_2) + 
%		B_2x_4x_6 + B_3\sigma_2+
%		B_4x_4\sigma_4- \dfrac{x_5}{\lvert x_5\rvert}B_5 + B_6\cos(x_2)\\
%		C_1x_4x_5 + 
%		C_2\sigma_3- \dfrac{x_6}{\lvert x_6\rvert}C_3
%	\end{bmatrix}
%\end{equation*} 
برای ساده‌سازی، ورودی مسئله را از سرعت دورانی به نیروهای تاثیرگذار در مودهای رول، پیچ و یاو تغیر داده شده‌است. این کار باعث می‌شود که مسئله از چند ورودی و چند خروجی به سه مسئله تک ورودی تبدیل شود. نیروها به فرم رابطه 
(\ref{SISO_force})
تعریف می‌شوند.
\begin{equation}\label{SISO_force}
	u_1 = \omega_2^2 - \omega_4^2, \quad
	u_2 = \omega_1^2 - \omega_3^2, \quad
	u_3 = \omega_1^2 - \omega_2^2  + \omega_3^2 - \omega_4^2
\end{equation}
با توجه به اینکه سه نیرو در نظر گرفته ‌شده و مسئله نیاز به چهار خروجی (سرعت دورانی موتورها) دارد یک نیروی دیگر نیز در نظر گرفته ‌می‌شود که به فرم رابطه 
(\ref{SISO_u4})
است و مقدار آن به‌صورت ثابت و برابر با سرعت دورانی تمام پره‌ها در دور نامی یعنی
\lr{2000 RPM}
در نظر گرفته ‌شده‌‌است.
\begin{equation}\label{SISO_u4}
	u_4 = \omega_1^2 + \omega_2^2  + \omega_3^2 + \omega_4^2
\end{equation}
در ادامه، روابط 
(\ref{SISO_force})
و
(\ref{SISO_u4})
را در فضای حالت سیستم جایگزین می‌کنیم و برای سادگی قسمت‌های 
\lr{$(\omega_1-\omega_2+\omega_3-\omega_4$)}
را از معادلات حذف می‌کنیم.

فضای حالت جدید:
\begin{equation}
	\boldsymbol{\mathrm{f}} = \begin{bmatrix}
		x_4 + x_5\sin(x_1)\tan(x_2) + x_6\cos(x_1)\tan(x_2)\\
		x_5\cos(x_1)- x_6\sin(x_1)\\
		(x_5\sin(x_1) + x_6\cos(x_1))\sec(x_2)\\
		A_1\cos(x_2)\sin(x_1) + 
		A_2x_5x_6 + A_3u_1
		\\
		B_1\sin(x_2) + 
		B_2x_4x_6 + B_3u_2\\
		C_1x_4x_5 + 
		C_2u_3
	\end{bmatrix}
\end{equation} 
بردار ورودی جدید به‌صورت زیر تعریف می‌شود.
%\begin{equation}
%	\boldsymbol{x} = \begin{bmatrix}
%		\phi& \theta & \psi & p& q& r
%	\end{bmatrix}^\mathrm{T}
%\end{equation}
\begin{equation}
	\boldsymbol{\mathrm{u}} = \begin{bmatrix}
		u_1&u_2&u_3&u_4
	\end{bmatrix}^\mathrm{T}
\end{equation}
برای خطی سازی از بسط تیلور استفاده شده‌است.
\begin{equation}
	\delta \dot{\boldsymbol{\mathrm{x}}} = \boldsymbol{\mathrm{A}}\delta \boldsymbol{\mathrm{x}} + \boldsymbol{\mathrm{B}}\delta \boldsymbol{\mathrm{u}}
\end{equation}
\begin{equation}
	\boldsymbol{\mathrm{x}^*} = \begin{bmatrix} % bold or vec???????????/
		0& 0 & 0 & 0& 0& 0
	\end{bmatrix}^\mathrm{T}
\end{equation}
\begin{equation}
	\boldsymbol{\mathrm{u}^*} = \begin{bmatrix}
		0&0&0&4\times2000^2
	\end{bmatrix}^\mathrm{T}
\end{equation}

\begin{equation}
	\boldsymbol{\mathrm{A}} = \left.\dfrac{\partial \boldsymbol{\mathrm{f}}}{\partial \boldsymbol{\mathrm{x}}}\right\vert_{\boldsymbol{\mathrm{x}^*}}
%	\begin{bmatrix}
%		\dfrac{\partial  a_1}{\partial  x_1}&
%		\dfrac{\partial  a_1}{\partial  x_2}&
%		\dfrac{\partial  a_1}{\partial  x_3}&
%		\dfrac{\partial  a_1}{\partial  x_4}&
%		\dfrac{\partial  a_1}{\partial  x_5}&
%		\dfrac{\partial  a_1}{\partial  x_6}
%		\\[1em]
%		\dfrac{\partial  a_2}{\partial  x_1}&
%		\dfrac{\partial  a_2}{\partial  x_2}&
%		\dfrac{\partial  a_2}{\partial  x_3}&
%		\dfrac{\partial  a_2}{\partial  x_4}&
%		\dfrac{\partial  a_2}{\partial  x_5}&
%		\dfrac{\partial  a_2}{\partial  x_6}
%		\\[1em]
%		\dfrac{\partial  a_3}{\partial  x_1}&
%		\dfrac{\partial  a_3}{\partial  x_2}&
%		\dfrac{\partial  a_3}{\partial  x_3}&
%		\dfrac{\partial  a_3}{\partial  x_4}&
%		\dfrac{\partial  a_3}{\partial  x_5}&
%		\dfrac{\partial  a_3}{\partial  x_6}
%		\\[1em]
%		\dfrac{\partial  a_4}{\partial  x_1}&
%		\dfrac{\partial  a_4}{\partial  x_2}&
%		\dfrac{\partial  a_4}{\partial  x_3}&
%		\dfrac{\partial  a_4}{\partial  x_4}&
%		\dfrac{\partial  a_4}{\partial  x_5}&
%		\dfrac{\partial  a_4}{\partial  x_6}
%		\\[1em]
%		\dfrac{\partial  a_5}{\partial  x_1}&
%		\dfrac{\partial  a_5}{\partial  x_2}&
%		\dfrac{\partial  a_5}{\partial  x_3}&
%		\dfrac{\partial  a_5}{\partial  x_4}&
%		\dfrac{\partial  a_5}{\partial  x_5}&
%		\dfrac{\partial  a_5}{\partial  x_6}
%		\\[1em]
%		\dfrac{\partial  a_6}{\partial  x_1}&
%		\dfrac{\partial  a_6}{\partial  x_2}&
%		\dfrac{\partial  a_6}{\partial  x_3}&
%		\dfrac{\partial  a_6}{\partial  x_4}&
%		\dfrac{\partial  a_6}{\partial  x_5}&
%		\dfrac{\partial  a_6}{\partial  x_6}
%	\end{bmatrix}
\end{equation}
\begin{equation}
	\boldsymbol{\mathrm{B}} = \left.\dfrac{\partial \boldsymbol{\mathrm{f}}}{\partial \boldsymbol{\mathrm{u}}}\right\vert_{\boldsymbol{\mathrm{u}^*}}
%	\begin{bmatrix}
%		\dfrac{\partial  a_1}{\partial  u_1}&
%		\dfrac{\partial  a_1}{\partial  u_2}&
%		\dfrac{\partial  a_1}{\partial  u_3}&
%		\dfrac{\partial  a_1}{\partial  u_4}
%		\\[1em]
%		\dfrac{\partial  a_2}{\partial  u_1}&
%		\dfrac{\partial  a_2}{\partial  u_2}&
%		\dfrac{\partial  a_2}{\partial  u_3}&
%		\dfrac{\partial  a_2}{\partial  u_4}
%		\\[1em]
%		\dfrac{\partial  a_3}{\partial  u_1}&
%		\dfrac{\partial  a_3}{\partial  u_2}&
%		\dfrac{\partial  a_3}{\partial  u_3}&
%		\dfrac{\partial  a_3}{\partial  u_4}
%		\\[1em]
%		\dfrac{\partial  a_4}{\partial  u_1}&
%		\dfrac{\partial  a_4}{\partial  u_2}&
%		\dfrac{\partial  a_4}{\partial  u_3}&
%		\dfrac{\partial  a_4}{\partial  u_4}
%		\\[1em]
%		\dfrac{\partial  a_5}{\partial  u_1}&
%		\dfrac{\partial  a_5}{\partial  u_2}&
%		\dfrac{\partial  a_5}{\partial  u_3}&
%		\dfrac{\partial  a_5}{\partial  u_4}
%		\\[1em]
%		\dfrac{\partial  a_6}{\partial  u_1}&
%		\dfrac{\partial  a_6}{\partial  u_2}&
%		\dfrac{\partial  a_6}{\partial  u_3}&
%		\dfrac{\partial  a_6}{\partial  u_4}
%		\\[1em]
%	\end{bmatrix}
\end{equation}
روابط بالا به فرم چند سیستم چند ورودی و چند خروجی نوشته ‌شده‌است. آن را به تک ورودی تبدیل می‌کنیم.
%\subsubsection{مود رول}
بنابراین، ماتریس‌های $\boldsymbol{A}$ و $\boldsymbol{B}$ کانال رول به‌صورت زیر است.
\begin{equation}
	\boldsymbol{\mathrm{A}}_{\text{roll}} = \begin{bmatrix}
		\dfrac{\partial  f_1}{\partial  x_1}& \dfrac{\partial  f_1}{\partial  x_4}
		\\[1em]
		\dfrac{\partial  f_4}{\partial  x_1}& \dfrac{\partial  f_4}{\partial  x_4}
	\end{bmatrix} = 
	\begin{bmatrix}
		0 & 1\\
		A_1\cos(x_1) & 0
	\end{bmatrix}
\end{equation}
\begin{equation}
	\boldsymbol{\mathrm{B}}_{\text{roll}}  = \begin{bmatrix}
		\dfrac{\partial  f_1}{\partial  u_1}
		\\[1em]
		\dfrac{\partial  f_4}{\partial  u_1}
	\end{bmatrix} = 
	\begin{bmatrix}
		0\\
		A_3
	\end{bmatrix}
\end{equation}

%\subsubsection{مود پیچ}
همچنین، ماتریس‌های $\boldsymbol{A}$ و $\boldsymbol{B}$ کانال پیچ به‌صورت زیر است.
\begin{equation}
	\boldsymbol{\mathrm{A}}_{\text{pitch}}  = \begin{bmatrix}
		\dfrac{\partial  f_2}{\partial  x_2}& \dfrac{\partial  f_2}{\partial  x_5}
		\\[1em]
		\dfrac{\partial  f_5}{\partial  x_2}& \dfrac{\partial  f_5}{\partial  x_5}
	\end{bmatrix} = 
	\begin{bmatrix}
		0 & 1\\
		B_1\cos(x_1) & 0
	\end{bmatrix}
\end{equation}
\begin{equation}
	\boldsymbol{\mathrm{B}}_{\text{pitch}}  = \begin{bmatrix}
		\dfrac{\partial  f_2}{\partial  u_2}
		\\[1em]
		\dfrac{\partial  f_5}{\partial  u_2}
	\end{bmatrix} = 
	\begin{bmatrix}
		0\\
		B_3
	\end{bmatrix}
\end{equation}
%\subsubsection{مود یاو}
همچنین، ماتریس‌های $\boldsymbol{A}$ و $\boldsymbol{B}$ کانال یاو به‌صورت زیر است.
\begin{equation}
	\boldsymbol{\mathrm{A}}_{\text{yaw}} = \begin{bmatrix}
		\dfrac{\partial  f_3}{\partial  x_3}& \dfrac{\partial  f_3}{\partial  x_6}
		\\[1em]
		\dfrac{\partial  f_6}{\partial  x_3}& \dfrac{\partial  f_6}{\partial  x_6}
	\end{bmatrix} = 
	\begin{bmatrix}
		0 & 1\\
		0 & 0
	\end{bmatrix}
\end{equation}
\begin{equation}
	\boldsymbol{\mathrm{B}}_{\text{yaw}} = \begin{bmatrix}
		\dfrac{\partial  f_3}{\partial  u_3}
		\\[1em]
		\dfrac{\partial  f_6}{\partial  u_3}
	\end{bmatrix} = 
	\begin{bmatrix}
		0\\
		C_2
	\end{bmatrix}
\end{equation}
\subsubsection{استخراج سرعت دورانی پره‌ها از نیروها}
چهار معادله و چهار مجهول به‌صورت زیر است.
\begin{align}\label{4eq4ans}
	\begin{split}
		u_1 &= \omega_2^2 - \omega_4^2\\
		u_2 &= \omega_1^2 - \omega_3^2\\
		u_3 &= \omega_1^2 - \omega_2^2  + \omega_3^2 - \omega_4^2\\
		u_4 &= \omega_1^2 + \omega_2^2  + \omega_3^2 + \omega_4^2
	\end{split}
\end{align}
جواب معادلات 
(\ref{4eq4ans})
به‌صورت رابطه 
(\ref{u2omega})
به دست می‌آید.
\begin{equation}\label{u2omega}
	\begin{split}
		\omega_1 &= \sqrt{\dfrac{u_4 + u_3 +2u_2}{4}}\\[1em]
		\omega_2 &= \sqrt{\dfrac{u_4 - u_3 +2u_1}{4}}\\[1em]
		\omega_3 &= \sqrt{\dfrac{u_4 + u_3 -2u_2}{4}}\\[1em]
		\omega_4 &= \sqrt{\dfrac{u_4 - u_3 -2u_1}{4}}
	\end{split}
\end{equation}
