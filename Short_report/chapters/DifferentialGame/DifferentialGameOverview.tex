\section{ مقدمه‌ای بر بازی دیفرانسیلی}\label{diffgameover}

 این پروژه حالت دو بازیکن را بررسی می‌کند. در این مسئله برای یک سامانه خطی پیوسته با معالات حالت:
 \begin{equation}\label{system_dynamic}
 	\begin{split}
 		 	&\boldsymbol{\dot{\mathrm{x}}}(t) = \boldsymbol{\mathrm{Ax}}(t) + \boldsymbol{\mathrm{B_1u_1}}(t) + \boldsymbol{\mathrm{B_2u_2}}(t)%, \quad \boldsymbol{x}(0) = \boldsymbol{x}_0%
 		\\
 		&\boldsymbol{\mathrm{y}}(t) = \boldsymbol{\mathrm{Cx}}(t) + \boldsymbol{\mathrm{D_1u_1}}(t) + \boldsymbol{\mathrm{D_2u_2}}(t)
 	\end{split}
 \end{equation}
که در رابطه (\ref{system_dynamic})
$\boldsymbol x$, $\boldsymbol y$, $\boldsymbol{u_1}$
و
$\boldsymbol{u_2}$
به ترتیب بیانگر بردار حالت، بردار خروجی، بردار ورودی بازیکن اول و بردار ورودی بازیکن دوم هستند. همچنین، 
$\boldsymbol A$, $\boldsymbol{B_1}$, $\boldsymbol {B_2}$, $\boldsymbol C$, $\boldsymbol {D_1}$
و
$\boldsymbol{D_2}$
به ترتیب بیانگر ماتریس حالت، ماتریس ورودی بازیکن اول، ماتریس ورودی بازیکن دوم، ماتریس خروجی، ماتریس فیدفوروارد بازیکن اول و ماتریس فیدفوروارد بازیکن دوم هستند
\cite{mct}.
بر اساس رابطه (\ref{system_dynamic}) دینامیک سامانه تحت تاثیر هر دو بازیکن قرار می‌گیرد. در اینجا ممکن است تلاش  بازیکن اول موجب دور شدن بازیکن دوم از هدف شود و یا برعکس.  این پروژه حالت همکاری دو بازیکن را بررسی نمی‌کند و دو بازیکن در تلاش برای کم‌کردن تابع هزینه خود و زیاد کردن تابع هزینه بازیکن مقابل هستند.


  فرض شده  که تابع هزینه برای هر بازیکن در زمان $t \in [0, T]$ به‌صورت مربعی\LTRfootnote{Quadratic Cost Function} است.
  هدف اصلی کم‌کردن تابع هزینه برای بازیکنان است. تابع هزینه برای بازیکن شماره \lr{i}  (این مسئله شامل دو بازیکن است) به فرم رابطه (\ref{cost}) نوشته می‌شود.

 \begin{equation}\label{cost}
 	J_i( \boldsymbol{\mathrm{u_1}},  \boldsymbol{\mathrm{u_2}}) = \int_{0}^{T}\left( \boldsymbol{\mathrm{x}} ^\mathrm{T}(t) \boldsymbol{\mathrm{Q_i}} \boldsymbol{\mathrm{x}}(t)+
 	 \boldsymbol{\mathrm{u_i}} ^\mathrm{T}(t) \boldsymbol{\mathrm{R_{ii}}} \boldsymbol{\mathrm{u_i}}(t)+
 	 \boldsymbol{\mathrm{u_j}} ^\mathrm{T}(t)\boldsymbol{\mathrm{ R_{ij} u_j}}(t)
 	\right)\mathrm{d}t
% 	\boldsymbol{ x} ^\mathrm{T}(T)\boldsymbol{ H_i}\boldsymbol{ x}(T) 
  \end{equation}
در رابطه
(\ref{cost})
$\boldsymbol{Q_i}$, $\boldsymbol{R_{ii}}$
و
$\boldsymbol{R_{ij}}$
به ترتیب بیانگر اهمیت میزان انحراف متغیرهای حالت مقادیر مطلوب برای بازیکن شماره \lr{i}، میزان تلاش کنترلی بازیکن شماره \lr{i} و میزان تلاش کنترلی بازیکن شماره \lr{j}  هستند.
در اینجا ماتریس‌های 
$\boldsymbol{Q_i}$، $\boldsymbol{R_{ii}}$
و
$\boldsymbol{H}$
متقارن فرض شده‌اند و ماتریس 
$\boldsymbol{R_{ii}}$
به‌صورت مثبت معین ($\boldsymbol{R_{ii}}>0$)
فرض شده‌است \cite{article1}.
%دینامیک سامانه تحت تاثیر هر دو بازیکن قرار می‌گیرد. در اینجا دینامیک سامانه به فرم رابطه (\ref{system_dynamic}) در نظر گرفته شده ‌است.

%در رابطه 
%(\ref{system_dynamic})،
%$\boldsymbol{u_1}$
%برابر با تلاش کنترلی  بازیکن اول است. در اینجا ممکن است تلاش  بازیکن اول موجب دور شدن بازیکن دوم از هدف شود و یا برعکس.  این پروژه حالت همکاری دو بازیکن را بررسی نمی‌کند و دو بازیکن در تلاش برای کم‌کردن تابع هزینه خود و زیاد کردن تابع هزینه بازیکن مقابل هستند.
در این حالت فرض شده‌است که تمامی بازیکنان در زمان 
$t \in [0, T]$
فقط اطلاعات شرایط اولیه و مدل سامانه را دارند. این فرض به این صورت تفسیر می‌شود که دو بازیکن همزمان حرکت خود را انتخاب می‌کنند. در این حالت امکان هماهنگی بین دو بازیکن وجود ندارد. تعادل نش یک راه حل برای بازی دیفرانسیلی با شرایط اشاره شده ارائه می‌دهد.
\شروع{قضیه} 
به مجموعه‌ای از حرکات قابل قبول 
$(\boldsymbol{u_1}^*,  \boldsymbol{u_2}^*)$
یک \مهم{تعادل نش} برای بازی می‌گویند اگر تمامی حرکات قابل قبول 
$(\boldsymbol{u_1},  \boldsymbol{u_2})$
از نامساوی (\ref{nash_lqe}) پیروی کنند.
\begin{equation}\label{nash_lqe}
	J_1(\boldsymbol{\mathrm{u_1}}^*, \boldsymbol{\mathrm{u_2}}^*)\leq J_1(\boldsymbol{\mathrm{u_1}}, \boldsymbol{\mathrm{u_2}}^*) \text{\rm{ and }}
	J_2(\boldsymbol{\mathrm{u_1}}^*, \boldsymbol{\mathrm{u_2}}^*)\leq 
	J_2(\boldsymbol{\mathrm{u_1}}^*, \boldsymbol{\mathrm{u_2}})
\end{equation}
\پایان{قضیه}
در اینجا قابل قبول بودن به‌معنی آن است که
$\boldsymbol{u_i}(.)$
به یک مجموعه محدود حرکات تعلق دارد، این مجموعه‌ی حرکات که بستگی به اطلاعات بازیکنان از بازی دارد، مجموعه‌ای از راهبردهایی است که بازیکنان ترجیح می‌دهند برای کنترل سامانه انجام دهند و سامانه 
(\ref{system_dynamic})
باید یک جواب منحصر به فرد داشته باشد. 


تعادل نش به گونه‌ای تعریف می‌شود که هیچ یک از بازیکنان انگیزه‌ی یک طرفه برای انحراف از بازی ندارند. قابل ذکر است که نمی‌توان انتظار داشت که یک تعادل نش منحصر به فرد وجود داشته باشد. 
%به هر حال به راحتی می‌توان تایید کرد که حرکات
%$(\boldsymbol{u_1}^*, \boldsymbol{u_2}^*)$
%یک تعادل نش برای بازی با تابع هزینه
%$J_i,~ (i = 1, 2)$
%است.